\section{Εισαγωγή}
Ο κώδικας όλων των παραδειγμάτων βρίσκεται στο \href{https://github.com/chgogos/ceteiep_dsa}{https://github.com/chgogos/ceteiep\_dsa}.

\section{Στοίβα}
Η στοίβα (stack) είναι μια ειδική περίπτωση γραμμικής λίστας στην οποία οι εισαγωγές και οι διαγραφές επιτρέπονται μόνο από το ένα άκρο. Συνήθως αυτό το άκρο λέγεται κορυφή (top). Πρόκειται για μια δομή στην οποία οι εισαγωγές και οι εξαγωγές γίνονται σύμφωνα με το μοντέλο τελευταίο μέσα πρώτο έξω (LIFO=Last In First Out).

Στη συνέχεια παρουσιάζεται μια υλοποίηση στοίβας που χρησιμοποιεί συνδεδεμένη λίστα της STL για την αποθήκευση των στοιχείων της.

\section{Ουρά}
Η ουρά (queue) είναι μια ειδική περίπτωση γραμμικής λίστας στην οποία επιτρέπονται εισαγωγές στο πίσω άκρο της και εξαγωγές από το εμπρός άκρο της μόνο. Τα δύο αυτά άκρα συνήθως αναφέρονται ως
πίσω (rear) και εμπρός (front) αντίστοιχα. Η ουρά είναι μια δομή στην οποία οι εισαγωγές και οι εξαγωγές γίνονται σύμφωνα με το μοντέλο πρώτο μέσα πρώτο έξω (FIFO=First In First Out).

Στη συνέχεια παρουσιάζεται μια υλοποίηση ουράς που χρησιμοποιεί συνδεδεμένη λίστα της STL για την αποθήκευση των στοιχείων της.

\section{Παραδείγματα}

\section{Ασκήσεις}
\begin{enumerate}
\item a
\item b
\item c
\item d
\end{enumerate}

\begin{thebibliography}{9}

\end{thebibliography}

