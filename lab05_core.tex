\section{Εισαγωγή}
Ο κώδικας όλων των παραδειγμάτων βρίσκεται στο \href{https://github.com/chgogos/ceteiep_dsa}{https://github.com/chgogos/ceteiep\_dsa}.

\section{Στοίβα}
Η στοίβα (stack) είναι μια ειδική περίπτωση γραμμικής λίστας στην οποία οι εισαγωγές και οι διαγραφές επιτρέπονται μόνο από το ένα άκρο. Συνήθως αυτό το άκρο λέγεται κορυφή (top). Πρόκειται για μια δομή στην οποία οι εισαγωγές και οι εξαγωγές γίνονται σύμφωνα με το μοντέλο τελευταίο μέσα πρώτο έξω (LIFO=Last In First Out).

Στη συνέχεια παρουσιάζεται μια υλοποίηση στοίβας.
\lstinputlisting[caption = Υλοποίηση στοίβας (stack\_oo.cpp),label=lst:stack_oo.cpp]{lab05/stack_oo.cpp}

\lstinputlisting[style=DOS]{lab05/stack_oo.out}

\section{Ουρά}
Η ουρά (queue) είναι μια ειδική περίπτωση γραμμικής λίστας στην οποία επιτρέπονται εισαγωγές στο πίσω άκρο της και εξαγωγές από το εμπρός άκρο της μόνο. Τα δύο αυτά άκρα συνήθως αναφέρονται ως
πίσω (rear) και εμπρός (front) αντίστοιχα. Η ουρά είναι μια δομή στην οποία οι εισαγωγές και οι εξαγωγές γίνονται σύμφωνα με το μοντέλο πρώτο μέσα πρώτο έξω (FIFO=First In First Out).

Στη συνέχεια παρουσιάζεται μια υλοποίηση ουράς που χρησιμοποιεί συνδεδεμένη λίστα της STL για την αποθήκευση των στοιχείων της.

\section{Οι δομές στοίβα και ουρά στην STL}
\subsection{std::stack}
Η std::stack έχει υλοποιηθεί στην STL ως ένα container adaptor δηλαδή ως μια κλάση που χρησιμοποιεί εσωτερικά ένα άλλο container και παρέχει ένα συγκεκριμένο σύνολο από λειτουργίες που επιτρέπουν την προσπέλαση και την τροποποίηση των στοιχείων του. Το εσωτερικό container μπορεί να είναι κάποιο από τα containers της STL: vector, list, dequeue ή οποιοδήποτε container που υποστηρίζει τις λειτουργίες: empty, size, back, push\_back και pop\_back. Τυπικές λειτουργίες που παρέχει η std::stack είναι οι ακόλουθες:
\begin{itemize}
\item empty, ελέγχει αν η στοίβα είναι άδεια.
\item size, επιστρέφει το μέγεθος της στοίβας.
\item top, προσπελαύνει το στοιχείο που βρίσκεται στη κορυφή της στοίβας (χωρίς να το αφαιρεί).
\item push, ωθεί ένα στοιχείο στη κορυφή της στοίβας
% \item emplace, δημιουργεί και εισάγει ένα στοιχείο στη κορυφή της στοίβας.
\item pop, αφαιρεί το στοιχείο που βρίσκεται στη κορυφή της στοίβας.
% \item swap, αντιμεταθέτει τα περιεχόμενα από δύο στοίβες.
\end{itemize}

\section{Παραδείγματα}

\section{Ασκήσεις}
\begin{enumerate}
\item a
\item b
\item c
\item d
\end{enumerate}

\begin{thebibliography}{9}

\end{thebibliography}

