\documentclass[11pt,a4paper]{book}

\usepackage{ceteiep_dsa_notes}

% \usepackage{draftwatermark}
% \SetWatermarkScale{1}

\addto\captionsgreek{\renewcommand{\chaptername}{Εργαστήριο}}

\title{Δομές Δεδομένων και Αλγόριθμοι \\ Εργαστήριο (C++)\\ Τ.Ε.Ι. Ηπείρου - Τμήμα Μηχανικών Πληροφορικής Τ.Ε. \\ Έκδοση 1.1}
\author{Χρήστος Γκόγκος  \\ Αναπληρωτής Καθηγητής }
\date{Χειμερινό εξάμηνο 2018-2019}

\begin{document}
\frontmatter
\maketitle
\tableofcontents
\mainmatter

% Εργαστήριο 1
\chapter{Βασικές έννοιες στη C και στη C++}
\section{Εισαγωγή}
Στο πρώτο αυτό εργαστήριο θα επιχειρηθεί μια παρουσίαση των βασικών γνώσεων που απαιτούνται έτσι ώστε να είναι δυνατή η κατανόηση των εργαστηρίων που ακολουθούν. Ειδικότερα, θα γίνει αναφορά σε δείκτες, στη δυναμική δέσμευση και αποδέσμευση μνήμης, στο πέρασμα παραμέτρων σε συναρτήσεις, στους πίνακες, στις δομές, στα αντικείμενα και τέλος στην ανάγνωση και εγγραφή σε αρχεία. Θα παρουσιαστούν λυμένα παραδείγματα καθώς και εκφωνήσεις ασκήσεων προς επίλυση. Αναλυτικότερη παρουσίαση των ανωτέρω θεμάτων γίνεται στα ελεύθερα διαθέσιμα βιβλία \cite{stamatiadis2017}, \cite{downey2012}, \cite{soulie2007}, \cite{hall2007} που παρατίθενται ως αναφορές στο τέλος του κειμένου του εργαστηρίου. Ο κώδικας όλων των παραδειγμάτων βρίσκεται στο \href{}{}.

\section{Δείκτες}
Κάθε θέση μνήμης στην οποία μπορούν να αποθηκευτούν δεδομένα βρίσκεται σε μια διεύθυνση μνήμης. Η δε μνήμη του υπολογιστή αποτελείται από ένα συνεχόμενο χώρο διευθύνσεων. Αν μια μεταβλητή δηλωθεί ως τύπου int * τότε η τιμή που θα λάβει ερμηνεύεται ως μια διεύθυνση που δείχνει σε μια θέση μνήμης η οποία περιέχει έναν ακέραιο. Από την άλλη μεριά το σύμβολο \& επιτρέπει τη λήψη της διεύθυνσης μιας μεταβλητής. Στον ακόλουθο κώδικα δηλώνονται 2 ακέραιες μεταβλητές (a και b) και ένας δείκτης (p) σε ακέραια τιμή. Ο δείκτης p λαμβάνει ως τιμή τη διεύθυνση της μεταβλητής a. Στη συνέχεια, οι μεταβλητές a και b λαμβάνουν τιμές μέσω του δείκτη p. Για να συμβεί αυτό γίνεται έμμεση αναφορά ή αλλιώς αποαναφορά (dereference) του δείκτη με το *p. Συνεπώς, το *p αντιστοιχεί στο περιεχόμενο της διεύθυνσης μνήμης που έχει ο δείκτης p.

\lstinputlisting[caption=Παράδειγμα με δείκτες (lab01\_01.cpp)]{lab01/lab01_01.cpp}
\lstinputlisting[style=DOS]{lab01/lab01_01.out}

Ένα συνηθισμένο λάθος με δείκτες παρουσιάζεται όταν γίνεται dereference ενός δείκτη (δηλαδή, δεδομένου ενός δείκτη p όταν χρησιμοποιείται το *p) χωρίς ο δείκτης να έχει αρχικοποιηθεί πρώτα δείχνοντας σε μια έγκυρη θέση μνήμης. Σε αυτή την περίπτωση το πρόγραμμα καταρρέει. 

\lstinputlisting[caption=Λανθασμένη χρήση δείκτη (lab01\_02.cpp)]{lab01/lab01_02.cpp}

\begin{lstlisting}[style=DOS]
segmentation fault
\end{lstlisting}

Αν η μεταγλώττιση του κώδικα γίνει με το flag -Wall τότε θα εμφανιστεί μήνυμα που θα προειδοποιεί για τη λάθος χρήση του δείκτη.

\begin{lstlisting}[style=DOS]
g++ -Wall lab01_02.cpp -o lab01_02
lab01_02.cpp: In function 'int main(int, char**)':
lab01_02.cpp:8:11: warning: 'p' is used uninitialized in this function [-Wuninitialized]
     *p = 2;
           ^
\end{lstlisting}

\section{Κλήση με τιμή και κλήση με αναφορά}
Οι δείκτες μπορούν να χρησιμοποιηθούν έτσι ώστε να επιτευχθεί, εφόσον απαιτείται, κλήση με αναφορά (call by reference) στις παραμέτρους μιας συνάρτησης και όχι κλήση με τιμή (call by value) που είναι ο προκαθορισμένος τρόπος κλήσης συναρτήσεων. Στο παράδειγμα που ακολουθεί η συνάρτηση swap (σε αντίθεση με τη συνάρτηση swap\_impotent) επιτυγχάνει την αντιμετάθεση των δύο μεταβλητών που δέχεται ως ορίσματα καθώς χρησιμοποιεί δείκτες που αναφέρονται στις ίδιες τις μεταβλητές του κυρίου προγράμματος και όχι σε αντίγραφά τους. 

\lstinputlisting[caption=Αντιμετάθεση μεταβλητών με δείκτες (lab01\_03.cpp)]{lab01/lab01_03.cpp}

\lstinputlisting[style=DOS]{lab01/lab01_03.out}


Η γλώσσα C++ προκειμένου να απλοποιήσει την κλήση με αναφορά εισήγαγε την έννοια των ψευδωνύμων (aliases). Τοποθετώντας στη δήλωση μιας παραμέτρου συνάρτησης το σύμβολο \& η παράμετρος λειτουργεί ως ψευδώνυμο για τη μεταβλητή που περνά στην αντίστοιχη θέση. Η συγκεκριμένη συμπεριφορά παρουσιάζεται στον ακόλουθο κώδικα.

\lstinputlisting[caption=Αντιμετάθεση μεταβλητών με αναφορές (lab01\_04.cpp)]{lab01/lab01_04.cpp}

\lstinputlisting[style=DOS]{lab01/lab01_04.out}


\section{Πίνακες}
Ένας  πίνακας είναι μια συλλογή από στοιχεία του ίδιου τύπου καθένα από τα οποία μπορεί να αναγνωριστεί από την τιμή ενός ακεραίου δείκτη (index).  Το γεγονός αυτό επιτρέπει την τυχαία προσπέλαση (random access) στα στοιχεία του πίνακα. Οι δείκτες των πινάκων ξεκινούν από το μηδέν. 

\subsection{Μονοδιάστατοι πίνακες}
Οι μονοδιάστατοι πίνακες είναι η πλέον απλή δομή δεδομένων. Η αναφορά στα στοιχεία του πίνακα γίνεται συνήθως με μια δομή επανάληψης (π.χ. for). Στο ακόλουθο παράδειγμα δύο μονοδιάστατοι πίνακες αρχικοποιούνται κατά τη δήλωσή τους και εν συνεχεία υπολογίζεται το εσωτερικό γινόμενό τους δηλαδή το άθροισμα των γινομένων των στοιχείων των πινάκων που βρίσκονται στην ίδια θέση.

\lstinputlisting[caption=Υπολογισμός εσωτερικού γινομένου δύο πινάκων (lab01\_05.cpp)]{lab01/lab01_05.cpp}

\lstinputlisting[style=DOS]{lab01/lab01_05.out}


\subsection{Δυναμικοί πίνακες}
Δυναμικοί πίνακες χρησιμοποιούνται όταν το μέγεθος του πίνακα πρέπει να αλλάζει κατά τη διάρκεια εκτέλεσης του προγράμματος και συνεπώς δεν μπορεί να ορισθεί κατά τη μεταγλώττιση. Πριν χρησιμοποιηθεί ένας δυναμικός πίνακας θα πρέπει δεσμευτούν οι απαιτούμενες θέσεις μνήμης. Επίσης, θα πρέπει να απελευθερωθεί ο χώρος που καταλαμβάνει όταν πλέον δεν χρησιμοποιείται. Στο ακόλουθο παράδειγμα ο χρήστης εισάγει το μέγεθος ενός μονοδιάστατου πίνακα και ο απαιτούμενος χώρος δεσμεύεται κατά την εκτέλεση του κώδικα. Στη συνέχεια ο πίνακας γεμίζει με τυχαίες ακέραιες τιμές στο διάστημα [1,100]. Παρουσιάζονται δύο εκδόσεις του κώδικα, μια που χρησιμοποιεί τις συναρτήσεις malloc και free της γλώσσας C και μια που χρησιμοποιεί τις εντολές new και delete της C++ για τη δέσμευση και την αποδέσμευση μνήμης. Επιπλέον, χρησιμοποιείται διαφορετικός τρόπος για τη δημιουργία των τυχαίων τιμών στα δύο προγράμματα.

\lstinputlisting[caption=Δημιουργία δυναμικού πίνακα με συναρτήσεις της C (lab01\_06.cpp)]{lab01/lab01_06.cpp}

\lstinputlisting[style=DOS]{lab01/lab01_06.out}

\lstinputlisting[caption=Δημιουργία δυναμικού πίνακα με συναρτήσεις της C++ (lab01\_07.cpp),label=lst:lab01_07.cpp]{lab01/lab01_07.cpp}
\lstinputlisting[style=DOS]{lab01/lab01_07.out}

Για να γίνει η μεταγλώττιση του κώδικα \ref{lst:lab01_07.cpp} θα πρέπει να χρησιμοποιηθεί το flag -std=c++11 όπως φαίνεται στην ακόλουθη εντολή.

\begin{lstlisting}[style=DOS]
g++ -std=c++11 lab01_07.cpp -o lab01_07
\end{lstlisting}

\subsection{Πίνακας ως παράμετρος συνάρτησης}
Ένας πίνακας μπορεί να περάσει ως παράμετρος σε μια συνάρτηση. Συχνά χρειάζεται να περάσουν ως παράμετροι και οι διαστάσεις του πίνακα. Στον ακόλουθο κώδικα η συνάρτηση simple\_stats δέχεται ως παράμετρο έναν μονοδιάστατο  πίνακα ακεραίων και το πλήθος των στοιχείων του και επιστρέφει μέσω κλήσεων με αναφορά το μέσο όρο, το ελάχιστο και το μέγιστο από όλα τα στοιχεία του πίνακα.

\lstinputlisting[caption=Δυναμικός πίνακας ως παράμετρος συνάρτησης (lab01\_08.cpp)]{lab01/lab01_08.cpp}

\lstinputlisting[style=DOS]{lab01/lab01_08.out}

\subsection{Δισδιάστατοι πίνακες}
Ένας δισδιάστατος πίνακας αποτελείται από γραμμές και στήλες και η αναφορά στα στοιχεία του γίνεται με δύο  δείκτες από τους οποίους ο πρώτος δείκτης υποδηλώνει τη γραμμή και ο δεύτερος υποδηλώνει τη στήλη του πίνακα. Οι πίνακες είναι ιδιαίτερα σημαντικοί για την εκτέλεση μαθηματικών υπολογισμών (π.χ. πολλαπλασιασμό πινάκων, επίλυση συστημάτων γραμμικών εξισώσεων κ.α.). Στον ακόλουθο κώδικα δίνεται ένα παράδειγμα δήλωσης ενός δισδιάστατου πίνακα 5 x 4 ο οποίος περνά ως παράμετρος στη συνάρτηση sums\_row\_wise. Η δε συνάρτηση επιστρέφει το άθροισμα κάθε γραμμής του πίνακα.

\lstinputlisting[caption=Δισδιάστατος πίνακας ως παράμετρος συνάρτησης (lab01\_09.cpp)]{lab01/lab01_09.cpp}

\lstinputlisting[style=DOS]{lab01/lab01_09.out}


\subsection{Πολυδιάστατοι πίνακες}
Αν και οι μονοδιάστατοι και οι δισδιάστατοι πίνακες χρησιμοποιούνται συχνότερα, υποστηρίζονται και πίνακες μεγαλύτερων διαστάσεων. Στη συνέχεια δίνεται ένα παράδειγμα δήλωσης και αρχικοποίησης ενός τρισδιάστατου πίνακα 3x3x2 και ενός τετραδιάστατου πίνακα 3x3x3x2.

\lstinputlisting[caption=Δήλωση και αρχικοποίηση τρισδιάστατου και τετραδιάστατου πίνακα (lab01\_10.cpp)]{lab01/lab01_10.cpp}


\subsection{Πριονωτοί πίνακες}
Εφόσον ένας πολυδιάστατος πίνακας δημιουργείται δυναμικά μπορεί να οριστεί με τέτοιο τρόπο έτσι ώστε η κάθε γραμμή του να μην έχει τον ίδιο αριθμό στοιχείων. Στον ακόλουθο κώδικα δημιουργείται ένας δισδιάστατος πίνακας 5 γραμμών με την πρώτη γραμμή να έχει 1 στοιχείο και κάθε επόμενη γραμμή ένα περισσότερο στοιχείο από την προηγούμενη της.

\lstinputlisting[caption=Παράδειγμα πριονωτού πίνακα με 5 γραμμές (lab01\_11.cpp)]{lab01/lab01_11.cpp}

\lstinputlisting[style=DOS]{lab01/lab01_11.out}

\section{Δομές}
Οι δομές χρησιμοποιούνται όταν απαιτούνται σύνθετοι τύποι δεδομένων οι οποίοι αποτελούνται από επιμέρους στοιχεία. Στο παράδειγμα που ακολουθεί ορίζεται η δομή Book με 3 πεδία. Στη συνέχεια δημιουργούνται 3 μεταβλητές που πρόκειται να αποθηκεύσουν πληροφορίες για ένα βιβλίο η κάθε μια.  Η τρίτη μεταβλητή είναι δείκτης προς τη δομή Book και προκειμένου να χρησιμοποιηθεί θα πρέπει πρώτα να δεσμευθεί μνήμη (new) ενώ με τον τερματισμό του προγράμματος θα πρέπει η μνήμη αυτή να επιστραφεί στο σύστημα (delete).

\lstinputlisting[caption=Μεταβλητές τύπου δομής Book (lab01\_12.cpp)]{lab01/lab01_12.cpp}

\lstinputlisting[style=DOS]{lab01/lab01_12.out}

\section{Κλάσεις - Αντικείμενα}
Ο αντικειμενοστρεφής προγραμματισμός εντοπίζει τα αντικείμενα που απαρτίζουν την εφαρμογή και τα συνδυάζει προκειμένου να επιτευχθεί η απαιτούμενη λειτουργικότητα. Για κάθε αντικείμενο γράφεται μια κλάση η οποία είναι υπεύθυνη για τη δημιουργία των επιμέρους στιγμιοτύπων (object instances). Κάθε αντικείμενο έχει μεταβλητές και συναρτήσεις οι οποίες μπορεί να είναι είτε ιδιωτικές (private) είτε δημόσιες (public) (είτε προστατευμένες-protected).  Τα ιδιωτικά μέλη χρησιμοποιούνται εντός της κλάσης που ορίζει το αντικείμενο ενώ τα δημόσια μπορούν να χρησιμοποιηθούν και από κώδικα εκτός της κλάσης. Στο ακόλουθο παράδειγμα ορίζεται η κλάση Box η οποία έχει 3 ιδιωτικά μέλη (length, width, height) και 1 δημόσιο μέλος,  τη συνάρτηση volume. Στη main δημιουργούνται με τη βοήθεια του κατασκευαστή (constructor) δύο αντικείμενα (στιγμιότυπα) της κλάσης Box και καλείται για καθένα από αυτά η δημόσια συνάρτηση μέλος της Box, volume. 

\lstinputlisting[caption=Παράδειγμα κλάσης Box (lab01\_13.cpp)]{lab01/lab01_13.cpp}

\lstinputlisting[style=DOS]{lab01/lab01_13.out}

\section{Αρχεία}
Συχνά χρειάζεται να αποθηκεύσουμε δεδομένα σε αρχεία ή να επεξεργαστούμε δεδομένα τα οποία βρίσκονται σε αρχεία. Ο ακόλουθος κώδικας πρώτα δημιουργεί έναν αρχείο με 100 τυχαίους ακεραίους στον τρέχοντα κατάλογο και στη συνέχεια ανοίγει το αρχείο και εμφανίζει τα στοιχεία του.

\subsection{Εγγραφή και ανάγνωση δεδομένων από αρχείο με συναρτήσεις της C}

\lstinputlisting[caption=Εγγραφή 100 ακέραιων αριθμητικών δεδομένων σε αρχείο και ανάγνωση τους από το ίδιο αρχείο (lab01\_14.cpp)]{lab01/lab01_14.cpp}

\begin{lstlisting}[style=DOS]
811 718 632 412 529 957 359 735 498 302 855 265 749 756 336 625 489 870 120 177 ...
\end{lstlisting}

\subsection{Εγγραφή και ανάγνωση δεδομένων από αρχείο με συναρτήσεις της C++}
Η C++ έχει προσθέσει νέους τρόπους με τους οποίους μπορεί να γίνει η αλληλεπίδραση με τα αρχεία. Ακολουθεί ένα παράδειγμα εγγραφής και ανάγνωσης δεδομένων από αρχείο με τη χρήση των fstream και sstream.

\lstinputlisting[label=lst:lab01_15.cpp,caption=Εγγραφή και ανάγνωση αλφαριθμητικών και ακεραίων από αρχείο (lab01\_15.cpp) (lab01\_15.cpp)]{lab01/lab01_15.cpp}

\lstinputlisting[style=DOS]{lab01/lab01_15.out}

\section{Παραδείγματα}
\subsection{Παράδειγμα 1}
Γράψτε κώδικα που να δημιουργεί μια δομή με όνομα Point και να έχει ως πεδία 2 double αριθμούς  (x και y) που υποδηλώνουν τις συντεταγμένες του σημείου στο καρτεσιανό επίπεδο. Δημιουργήστε έναν πίνακα με όνομα points με 5 σημεία με απευθείας εισαγωγή τιμών για τα ακόλουθα σημεία: (4, 17), (10, 21), (5, 32), (-1, 16), (-4, 7). Γράψτε τον κώδικα που εμφανίζει τα 2 πλησιέστερα σημεία. Ποια είναι τα πλησιέστερα σημεία και ποια η απόσταση μεταξύ τους;

\lstinputlisting[caption=Λύση παραδείγματος 1 (lab01\_16.cpp)]{lab01/lab01_16.cpp}

\lstinputlisting[style=DOS]{lab01/lab01_16.out}

\subsection{Παράδειγμα 2}
Με τη γεννήτρια τυχαίων αριθμών mt19937 δημιουργήστε 10000 τυχαίες ακέραιες τιμές στο διάστημα 0 έως 10000 με seed την τιμή 1729. Τοποθετήστε τις τιμές σε ένα δισδιάστατο πίνακα 100 x 100 έτσι ώστε να συμπληρώνονται οι τιμές στον πίνακα κατά σειρές από πάνω προς τα κάτω και από αριστερά προς τα δεξιά. Να υπολογιστεί το άθροισμα της κάθε γραμμής του πίνακα. Ποιος είναι ο αριθμός της γραμμής με το μεγαλύτερο άθροισμα και ποιο είναι αυτό;

\lstinputlisting[caption=Λύση παραδείγματος 2 (lab01\_17.cpp)]{lab01/lab01_17.cpp}

\lstinputlisting[style=DOS]{lab01/lab01_17.out}

\subsection{Παράδειγμα 3}
Γράψτε 10000 τυχαίες ακέραιες τιμές στο διάστημα [1,10000] στο αρχείο data\_int\_10000.txt χρησιμοποιώντας τις συναρτήσεις rand και srand και seed την τιμή 1729. Διαβάστε τις τιμές από το αρχείο. Εντοπίστε τη μεγαλύτερη τιμή στα δεδομένα. Ποιες είναι οι τιμές  που εμφανίζονται τις περισσότερες φορές στα δεδομένα;

\lstinputlisting[caption=Λύση παραδείγματος 3 (lab01\_18.cpp)]{lab01/lab01_18.cpp}

\lstinputlisting[style=DOS]{lab01/lab01_18.out}
 
\subsection{Παράδειγμα 4}
Γράψτε κώδικα που να δημιουργεί μια δομή με όνομα student (σπουδαστής) και να έχει ως πεδία το name (όνομα) τύπου string και το grade (βαθμός) τύπου int. Διαβάστε τα περιεχόμενα του αρχείου που έχει δημιουργηθεί με τον κώδικα \ref{lst:lab01_15.cpp} (data\_student\_struct10.txt) και τοποθετήστε τα σε κατάλληλο πίνακα. Βρείτε τα ονόματα και το μέσο όρο βαθμολογίας των σπουδαστών με βαθμό άνω του μέσου όρου όλων των σπουδαστών. Θεωρείστε ότι οι βαθμοί έχουν αποθηκευτεί στο αρχείο data\_student\_struct10.txt ως ακέραιοι αριθμοί από το 0 μέχρι και το 100, αλλά η εμφάνισή τους θα πρέπει να γίνεται εφόσον πρώτα διαιρεθούν με το 10. Δηλαδή, ο βαθμός 55 αντιστοιχεί στο βαθμό 5.5.

\lstinputlisting[caption=Λύση παραδείγματος 4 (lab01\_19.cpp)]{lab01/lab01_19.cpp}

\lstinputlisting[style=DOS]{lab01/lab01_19.out}

\section{Ασκήσεις}
\begin{enumerate}
\item Γράψτε μια συνάρτηση που να δέχεται έναν πίνακα ακεραίων και το μέγεθός του και να επιστρέφει το μέσο όρο των τιμών καθώς και το πλήθος των τιμών που απέχουν το πολύ 10\% από το μέσο όρο. Δοκιμάστε την κλήση της συνάρτησης για έναν πίνακα 100 θέσεων με τυχαίες ακέραιες τιμές στο διάστημα [1,100] οι οποίες θα δημιουργηθούν με τη χρήση των συναρτήσεων srand() και rand() της C. Χρησιμοποιήστε ως seed για την αρχικοποίηση των τυχαίων τιμών την τιμή 12345.

\item Γράψτε πρόγραμμα που να διαβάζει τα στοιχεία υπαλλήλων (όνομα, μισθό και έτη προϋπηρεσίας) από το αρχείο data\_ypallhlos\_struct20.txt και να εμφανίζει τα στοιχεία του κάθε υπαλλήλου μέσω μιας συνάρτησης που θα δέχεται ως παράμετρο μια μεταβλητή τύπου δομής υπαλλήλου. Στη συνέχεια να υπολογίζει και να εμφανίζει το ποσό που θα συγκεντρωθεί αν για κάθε υπάλληλο με περισσότερα από 5 έτη προϋπηρεσίας παρακρατηθεί το 5\% του μισθού του ενώ για τους υπόλοιπους υπαλλήλους παρακρατηθεί το 7\% του μισθού τους.

\item Γράψτε το προηγούμενο πρόγραμμα ξανά χρησιμοποιώντας κλάση στη θέση της δομής. Επιπλέον ορίστε constructor και getters/setters για τα μέλη δεδομένων του αντικειμένου υπάλληλος.

\item Γράψτε ένα πρόγραμμα που να γεμίζει έναν πίνακα a, 5 γραμμών και 5 στηλών, με τυχαίες ακέραιες τιμές στο διάστημα 1 έως και 1000 (χρησιμοποιήστε ως seed την τιμή 12345). Γράψτε μια συνάρτηση που να δέχεται ως παράμετρο τον πίνακα a και να επιστρέφει σε μονοδιάστατο πίνακα col το άθροισμα των τιμών κάθε στήλης του πίνακα. Οι τιμές που επιστρέφονται να εμφανίζονται στο κύριο πρόγραμμα το οποίο να εμφανίζει επιπλέον και τον αριθμό στήλης με το μεγαλύτερο άθροισμα.
\end{enumerate}



\begin{thebibliography}{9}
\bibitem{stamatiadis2017}
Σταμάτης Σταματιάδης. Εισαγωγή στη γλώσσα προγραμματισμού C++11. Τμήμα Επιστήμης και Τεχνολογίας Υλικών, Πανεπιστήμιο Κρήτης, 2017, \href{https://www.materials.uoc.gr/el/undergrad/courses/ETY215/notes.pdf}{https://www.materials.uoc.gr/el/undergrad/courses/ETY215/notes.pdf}.

\bibitem{downey2012}
Allen B. Downey. How to think like a computer scientist, C++ version, 2012, \href{http://www.greenteapress.com/thinkcpp/}{http://www.greenteapress.com/thinkcpp/}. 

\bibitem{soulie2007}
Juan Souli\'e. C++ Language Tutorial. cplusplus.com, 2007, \href{http://www.cplusplus.com/files/tutorial.pdf}{http://www.cplusplus.com/files/tutorial.pdf}.

\bibitem{hall2007}
Brian Hall. Beej's Guide to C Programming, 2007, \href{http://beej.us/guide/bgc/}{http://beej.us/guide/bgc/}.

\end{thebibliography}


% Εργαστήριο 2
\chapter{Εισαγωγή στα templates, στην STL και στα lambdas - TDD}
\chaptermark{Templates, STL, lambdas, TDD}
\section{Εισαγωγή}
Στο εργαστήριο αυτό παρουσιάζεται ο μηχανισμός των templates και οι βασικές δυνατότητες της βιβλιοθήκης STL (Standard Template Library) της C++. Τα templates επιτρέπουν την κατασκευή γενικού κώδικα επιτρέποντας την αποτύπωση της λογικής μιας συνάρτησης ανεξάρτητα από τον τύπο των ορισμάτων που δέχεται. Από την άλλη μεριά, η βιβλιοθήκη STL, στην οποία γίνεται εκτεταμένη χρήση των templates παρέχει στον προγραμματιστή έτοιμη λειτουργικότητα για πολλές ενέργειες που συχνά συναντώνται κατά την ανάπτυξη εφαρμογών. Τέλος, γίνεται μια σύντομη αναφορά στην τεχνική ανάπτυξης προγραμμάτων TDD η οποία εξασφαλίζει σε κάποιο βαθμό την ανάπτυξη προγραμμάτων με ορθή λειτουργία εξαναγκάζοντας τους προγραμματιστές να ενσωματώσουν τη δημιουργία ελέγχων στον κώδικα που παράγουν καθημερινά. Επιπλέον υλικό για την STL βρίσκεται στις αναφορές \cite{stamatiadis2017}, \cite{geeks4geeks}, \cite{topcoder}, \cite{hackerearth}.

\section{Templates}
Τα templates είναι ένας μηχανισμός της C++ ο οποίος μπορεί να διευκολύνει τον προγραμματισμό. Η γλώσσα C++ είναι strongly typed και αυτό μπορεί να οδηγήσει στην ανάγκη υλοποίησης διαφορετικών εκδόσεων μιας συνάρτησης έτσι ώστε να υποστηριχθεί η ίδια λογική για διαφορετικούς τύπους δεδομένων. Για παράδειγμα, η εύρεση της ελάχιστης τιμής ανάμεσα σε τρεις τιμές θα έπρεπε να υλοποιηθεί με δύο συναρτήσεις έτσι ώστε να υποστηρίζει τόσο τους ακέραιους όσο και τους πραγματικούς αριθμούς, όπως φαίνεται στον κώδικα που ακολουθεί.

\lstinputlisting[caption=Επανάληψη λογικής κώδικα (lab02\_01.cpp)]{lab02/lab02_01.cpp}

\lstinputlisting[style=DOS]{lab02/lab02_01.out}

Με τη χρήση των templates μπορεί να γραφεί κώδικας που να υποστηρίζει ταυτόχρονα πολλούς τύπους δεδομένων.  Ειδικότερα, χρησιμοποιείται, η δεσμευμένη λέξη template και εντός των συμβόλων < και > τοποθετείται η λίστα των παραμέτρων του template. Ο μεταγλωττιστής αναλαμβάνει να δημιουργήσει όλες τις απαιτούμενες παραλλαγές των συναρτήσεων που θα χρειαστούν στον κώδικα που μεταγλωττίζει.

\lstinputlisting[caption=Χρήση template για αποφυγή επανάληψης λογικής κώδικα (lab02\_02.cpp)]{lab02/lab02_02.cpp}

\lstinputlisting[style=DOS]{lab02/lab02_02.out}

\section{Η βιβλιοθήκη STL}
Η βιβλιοθήκη STL (Standard Template Library) της C++ προσφέρει έτοιμη λειτουργικότητα για πολλά θέματα τα οποία ανακύπτουν συχνά στον προγραμματισμό εφαρμογών. Πρόκειται για μια generic βιβλιοθήκη, δηλαδή κάνει εκτεταμένη χρήση των templates. Βασικά τμήματα της STL είναι οι containers, οι iterators και οι αλγόριθμοι.

\subsection{Containers}
H STL υποστηρίζει έναν αριθμό από containers στους οποίους μπορούν να αποθηκευτούν δεδομένα. Ένα από τα containers είναι το vector. Στον ακόλουθο κώδικα φαίνεται πως η χρήση του vector διευκολύνει τον προγραμματισμό καθώς δεν απαιτούνται εντολές διαχείρισης μνήμης ενώ η δομή είναι δυναμική, δηλαδή το μέγεθος της μπορεί να μεταβάλλεται κατά τη διάρκεια εκτέλεσης του προγράμματος. 

\lstinputlisting[caption=Παράδειγμα με προσθήκη στοιχείων σε vector (lab02\_03.cpp)]{lab02/lab02_03.cpp}

\lstinputlisting[style=DOS]{lab02/lab02_03.out}

Ειδικότερα, τα containers χωρίζονται σε σειριακά (sequence containers) και συσχετιστικά (associate containers). Τα σειριακά containers είναι συλλογές ομοειδών στοιχείων στις οποίες κάθε στοιχείο  έχει συγκεκριμένη θέση μέσω της οποίας μπορούμε να αναφερθούμε σε αυτό. Τα σειριακά containers είναι τα εξής: 
\begin{itemize}[noitemsep]
\item array 
\item deque
\item forward\_list 
\item list 
\item vector
\end{itemize}

Τα συσχετιστικά containers παρουσιάζουν το πλεονέκτημα της γρήγορης προσπέλασης. Συσχετιστικά containers της STL είναι τα εξής: 
\begin{itemize}[noitemsep]
\item set 
\item multiset
\item map
\item multimap
\item unordered\_set
\item unordered\_multiset
\item unordered\_map 
\item unordered\_multimap. 
\end{itemize}

\subsection{Iterators}
Οι iterators αποτελούν γενικεύσεις των δεικτών και επιτρέπουν την πλοήγηση στα στοιχεία ενός container με τέτοιο τρόπο έτσι ώστε να μπορούν να χρησιμοποιηθούν οι ίδιοι αλγόριθμοι σε περισσότερα του ενός containers. Στον ακόλουθο κώδικα παρουσιάζεται το πέρασμα από τα στοιχεία ενός vector με 4  τρόπους. Καθώς το container είναι τύπου vector παρουσιάζεται αρχικά το πέρασμα από τις τιμές του με τη χρήση δεικτοδότησης τύπου πίνακα. Στη συνέχεια χρησιμοποιείται η πρόσβαση στα στοιχεία του container μέσω του range for. Ακολούθως, χρησιμοποιείται ένας iterator για πέρασμα από την αρχή προς το τέλος και μετά ένας reverse\_iterator για πέρασμα από το τέλος προς την αρχή. 

\lstinputlisting[caption = 4 διαφορετικοί τρόποι προσπέλασης των στοιχείων ενός vector (lab02\_04.cpp)]{lab02/lab02_04.cpp}

\lstinputlisting[style=DOS]{lab02/lab02_04.out}

\subsection{Αλγόριθμοι}
H STL διαθέτει πληθώρα αλγορίθμων που μπορούν να εφαρμοστούν σε διάφορα προβλήματα. Για παράδειγμα, προκειμένου να ταξινομηθούν δεδομένα μπορεί να χρησιμοποιηθεί η συνάρτηση sort της STL η οποία υλοποιεί τον αλγόριθμο \href{https://xlinux.nist.gov/dads/HTML/introspectiveSort.html}{Introspective Sort}. Στον ακόλουθο κώδικα πραγματοποιείται η ταξινόμηση ενός στατικού πίνακα και ενός vector.

\lstinputlisting[caption = Ταξινόμηση με τη συνάρτηση sort της STL (lab02\_05.cpp)]{lab02/lab02_05.cpp}

\lstinputlisting[style=DOS]{lab02/lab02_05.out}

Η συνάρτηση sort() εφαρμόζεται σε sequence containers πλην των list και forward\_list στα οποία δεν μπορεί να γίνει απευθείας πρόσβαση σε κάποιο στοιχείο με τη χρήση δείκτη. Ειδικά για αυτά τα containers υπάρχει συνάρτηση μέλος sort που επιτρέπει την ταξινόμησή τους. Στον ακόλουθο κώδικα δημιουργείται μια λίστα με αντικείμενα ορθογώνιων παραλληλογράμμων τα οποία ταξινομούνται με βάση το εμβαδόν τους σε φθίνουσα σειρά. Για την ταξινόμησή τους παρουσιάζονται 4 διαφορετικοί τρόποι.

\lstinputlisting[caption = Ταξινόμηση λίστας με αντικείμενα (lab02\_06.cpp)]{lab02/lab02_06.cpp}

\lstinputlisting[style=DOS]{lab02/lab02_06.out}


Αντίστοιχα, για να γίνει αναζήτηση ενός στοιχείου σε έναν ταξινομημένο πίνακα μπορούν να χρησιμοποιηθούν διάφορες συναρτήσεις της STL όπως η συνάρτηση binary\_search, ή η συνάρτηση upper\_bound. Ένα παράδειγμα χρήσης των συναρτήσεων αυτών δίνεται στον ακόλουθο κώδικα.

\lstinputlisting[caption = Αναζήτηση σε ταξινομημένο πίνακα (lab02\_07.cpp)]{lab02/lab02_07.cpp}

\lstinputlisting[style=DOS]{lab02/lab02_07.out}

\section{TDD=Test Driven Development}
Τα τελευταία χρόνια έχει αναγνωριστεί ως μια αποδοτική τεχνική ανάπτυξης εφαρμογών η οδηγούμενη από ελέγχους ανάπτυξη (Test Driven Development). Αν και το θέμα είναι αρκετά σύνθετο, η βασική ιδέα είναι ότι ο προγραμματιστής πρώτα γράφει κώδικα που ελέγχει αν η ζητούμενη λειτουργικότητα ικανοποιείται και στη συνέχεια προσθέτει τον κώδικα που θα υλοποιεί αυτή τη λειτουργικότητα. Ανά πάσα στιγμή υπάρχει ένα σύνολο από συσσωρευμένους ελέγχους οι οποίοι για κάθε αλλαγή που γίνεται στον κώδικα είναι σε θέση να εκτελεστούν άμεσα και να δώσουν εμπιστοσύνη μέχρι ένα βαθμό στο ότι το υπό κατασκευή ή υπό τροποποίηση λογισμικό λειτουργεί ορθά. Γλώσσες όπως η Java και η Python διαθέτουν εύχρηστες βιβλιοθήκες που υποστηρίζουν την ανάπτυξη TDD (junit και pytest αντίστοιχα). Στην περίπτωση της C++ επίσης υπάρχουν διάφορες βιβλιοθήκες που μπορούν να υποστηρίξουν την ανάπτυξη TDD. Στα πλαίσια του εργαστηρίου θα χρησιμοποιηθεί η βιβλιοθήκη \href{https://github.com/philsquared/Catch}{Catch} για το σκοπό της επίδειξης του TDD.

Στο παράδειγμα που ακολουθεί παρουσιάζεται η υλοποίηση της συνάρτησης παραγοντικό. Το παραγοντικό συμβολίζεται με το θαυμαστικό (!), ορίζεται μόνο για τους μη αρνητικούς ακεραίους αριθμούς και είναι το γινόμενο όλων των θετικών ακεραίων που είναι μικρότεροι ή ίσοι του αριθμού για τον οποίο ζητείται το παραγοντικό. Η πρώτη υλοποίηση είναι λανθασμένη καθώς δεν υπολογίζει σωστά το παραγοντικό του μηδενός που πρέπει να είναι μονάδα. 

\lstinputlisting[caption = Πρώτη έκδοση της συνάρτησης παραγοντικού και έλεγχοι (lab02\_08.cpp)]{lab02/lab02_08.cpp}

\lstinputlisting[style=DOS]{lab02/lab02_08.out}

Η δεύτερη υλοποίηση είναι σωστή. Τα μηνύματα που εμφανίζονται σε κάθε περίπτωση υποδεικνύουν το σημείο στο οποίο βρίσκεται το πρόβλημα και ότι πλέον αυτό λύθηκε.

\lstinputlisting[caption = Δεύτερη έκδοση της συνάρτησης παραγοντικού και έλεγχοι  (lab02\_09.cpp)]{lab02/lab02_09.cpp}

\lstinputlisting[style=DOS]{lab02/lab02_09.out}


\section{Παραδείγματα}
\subsection{Παράδειγμα 1}
Να γράψετε πρόγραμμα που να δημιουργεί πίνακα Α με 1.000 τυχαίες ακέραιες τιμές στο διάστημα [1, 10.000] και πίνακα Β με 100.000 τυχαίες ακέραιες τιμές στο ίδιο διάστημα τιμών. Η παραγωγή των τυχαίων τιμών να γίνει με τη γεννήτρια τυχαίων αριθμών mt19937 και με seed την τιμή 1821. Χρησιμοποιώντας τη συνάρτηση binary\_search της STL να βρεθεί πόσες από τις τιμές του Β υπάρχουν στον πίνακα Α.

\lstinputlisting[caption = Λύση παραδείγματος 1 (lab02\_10.cpp)]{lab02/lab02_10.cpp}

\lstinputlisting[style=DOS]{lab02/lab02_10.out}


\subsection{Παράδειγμα 2}
Η συνάρτηση accumulate() της STL επιτρέπει τον υπολογισμό αθροισμάτων στα στοιχεία ενός container. Δημιουργήστε ένα vector με τυχαίες ακέραιες τιμές και υπολογίστε το άθροισμα των τιμών του με τη χρήση της συνάρτησης accumulate. Επαναλάβετε τη διαδικασία για ένα container τύπου array.

\lstinputlisting[caption = Λύση παραδείγματος 2 (lab02\_11.cpp)]{lab02/lab02_11.cpp}

\lstinputlisting[style=DOS]{lab02/lab02_11.out}

\subsection{Παράδειγμα 3}
Δημιουργήστε ένα vector που να περιέχει ονόματα. Χρησιμοποιώντας τη συνάρτηση next\_permutation() εμφανίστε όλες τις διαφορετικές διατάξεις των ονομάτων που περιέχει το vector.

\lstinputlisting[caption = Λύση παραδείγματος 3 (lab02\_12.cpp)]{lab02/lab02_12.cpp}

\lstinputlisting[style=DOS]{lab02/lab02_12.out}

%\subsection{Παράδειγμα 4}
%
%\section{Ασκήσεις}
%\begin{enumerate}
%\item a
%\item a
%\item a
%\item a
%\end{enumerate}

\begin{thebibliography}{9}
\bibitem{stamatiadis2017}
Σταμάτης Σταματιάδης. Εισαγωγή στη γλώσσα προγραμματισμού C++11. Τμήμα Επιστήμης και Τεχνολογίας Υλικών, Πανεπιστήμιο Κρήτης, 2017, \href{https://www.materials.uoc.gr/el/undergrad/courses/ETY215/notes.pdf}{https://www.materials.uoc.gr/el/undergrad/courses/ETY215/notes.pdf}.

\bibitem{geeks4geeks}
\href{http://www.geeksforgeeks.org/cpp-stl-tutorial/}{http://www.geeksforgeeks.org/cpp-stl-tutorial/}.

\bibitem{topcoder}
\href{https://www.topcoder.com/community/data-science/data-science-tutorials/power-up-c-with-the-standard-template-library-part-1/}{https://www.topcoder.com/community/data-science/data-science-tutorials/power-up-c-with-the-standard-template-library-part-1/}.

\bibitem{hackerearth}
\href{https://www.hackerearth.com/practice/notes/standard-template-library/}{https://www.hackerearth.com/practice/notes/standard-template-library/}


\end{thebibliography}



% Εργαστήριο 3
\chapter{Θεωρητική μελέτη αλγορίθμων, χρονομέτρηση κώδικα, αλγόριθμοι ταξινόμησης και αλγόριθμοι αναζήτησης}
\chaptermark{Αλγόριθμοι ταξινόμησης και αλγόριθμοι αναζήτησης}
\section{Εισαγωγή}
Στο εργαστήριο αυτό παρουσιάζονται αλγόριθμοι ταξινόμησης και αναζήτησης. Πρόκειται για μερικούς από τους σημαντικότερους αλγορίθμους στην επιστήμη των υπολογιστών. Σε πρακτικό επίπεδο ένα σημαντικό ποσοστό της επεξεργαστικής ισχύος των υπολογιστών δαπανάται στην ταξινόμηση δεδομένων η οποία διευκολύνει τις αναζητήσεις που ακολουθούν. 
  

\section{Εκτίμηση και μέτρηση του χρόνου εκτέλεσης κώδικα}
Η απόδοση ενός αλγορίθμου μπορεί να εκτιμηθεί θεωρητικά και εμπειρικά. Η θεωρητική μελέτη προσδιορίζει την ασυμπτωτική συμπεριφορά του αλγορίθμου, δηλαδή πως θα συμπεριφέρεται ο αλγόριθμος καθώς τα δεδομένα εισόδου αυξάνονται σε μέγεθος. Με αυτό τον τρόπο μπορεί να συγκριθούν οι αποδόσεις αλγορίθμων που επιτελούν το ίδιο έργο. Για παράδειγμα ένας αλγόριθμος με χρονική πολυπλοκότητα $O(nlog(n))$ αναμένεται να έχει ταχύτερο χρόνο εκτέλεσης καθώς το μέγεθος των δεδομένων εισόδου αυξάνεται από έναν αλγόριθμο με χρονική πολυπλοκότητα $O(n^2)$. Από την άλλη μεριά η εμπειρική εκτίμηση της απόδοσης ενός προγράμματος έχει να κάνει με τη χρονομέτρησή του για διάφορες περιπτώσεις δεδομένων εισόδου και τη σύγκρισή του με εναλλακτικές υλοποιήσεις προγραμμάτων. Στη συνέχεια θα παρουσιαστούν δύο τρόποι μέτρησης χρόνου εκτέλεσης κώδικα που μπορούν να εφαρμοστούν στη C++.

\subsection{Μέτρηση χρόνου εκτέλεσης κώδικα με τη συνάρτηση clock()}
Ο ακόλουθος κώδικας μετράει το χρόνο που απαιτεί ο υπολογισμός του αθροίσματος των τετραγωνικών ριζών 10.000.000 τυχαίων ακέραιων αριθμών με τιμές στο διάστημα από 0  έως 10.000. Η μέτρηση του χρόνου πραγματοποιείται με τη συνάρτηση clock() η οποία επιστρέφει τον αριθμό από clock ticks που έχουν περάσει από τη στιγμή που το πρόγραμμα ξεκίνησε την εκτέλεση του. Ο αριθμός των δευτερολέπτων που έχουν περάσει προκύπτει διαιρώντας τον αριθμό των clock ticks με τη σταθερά CLOCKS\_PER\_SEC. Αυτός ο τρόπος υπολογισμού του χρόνου εκτέλεσης έχει ``κληρονομηθεί'' στη C++ από τη C.

\lstinputlisting[caption = Μέτρηση χρόνου εκτέλεσης κώδικα(timing1.cpp)]{lab03/timing1.cpp}

\lstinputlisting[style=DOS]{lab03/timing1.out}

\subsection{Μέτρηση χρόνου εκτέλεσης κώδικα με τη χρήση του high\_resolution\_clock::time\_point}
Η C++ έχει προσθέσει νέους τρόπους μέτρησης του χρόνου εκτέλεσης προγραμμάτων. Στον ακόλουθο κώδικα παρουσιάζεται ένα παράδειγμα με χρήση time\_points.

\lstinputlisting[caption = Μέτρηση χρόνου εκτέλεσης κώδικα (timing2.cpp)]{lab03/timing2.cpp}

\lstinputlisting[style=DOS]{lab03/timing2.out}


\section{Αλγόριθμοι ταξινόμησης}
\subsection{Ταξινόμηση με εισαγωγή}
Η ταξινόμηση με εισαγωγή (\href{http://rosettacode.org/wiki/Sorting_algorithms/Insertion_sort}{insertion-sort}) λειτουργεί δημιουργώντας μια ταξινομημένη λίστα στο αριστερό άκρο των δεδομένων και επαναληπτικά τοποθετεί το στοιχείο το οποίο βρίσκεται δεξιά της ταξινομημένης λίστας στη σωστή θέση σε σχέση με τα ήδη ταξινομημένα στοιχεία. Ο αλγόριθμος ταξινόμησης με εισαγωγή καθώς και η κλήση του από κύριο πρόγραμμα για την αύξουσα ταξινόμηση ενός πίνακα 10 θέσεων παρουσιάζεται στον κώδικα που ακολουθεί.

\lstinputlisting[caption = Ο αλγόριθμος ταξινόμησης με εισαγωγή (insertion\_sort.cpp)]{lab03/insertion_sort.cpp}

\lstinputlisting[caption = sort1.cpp]{lab03/sort1.cpp}

\lstinputlisting[style=DOS]{lab03/sort1.out}

\subsection{Ταξινόμηση με συγχώνευση}
Η ταξινόμηση με συγχώνευση (merge-sort) είναι αναδρομικός αλγόριθμος και στηρίζεται στη συγχώνευση ταξινομημένων υποακολουθιών έτσι ώστε να δημιουργούνται νέες ταξινομημένες υποακολουθίες. Μια υλοποίηση του κώδικα ταξινόμησης με συγχώνευση παρουσιάζεται στη συνέχεια. 

\lstinputlisting[caption = Ο αλγόριθμος ταξινόμησης με συγχώνευση (merge\_sort.cpp)]{lab03/merge_sort.cpp}

\lstinputlisting[caption = sort2.cpp]{lab03/sort2.cpp}

\lstinputlisting[style=DOS]{lab03/sort2.out}

\subsection{Γρήγορη ταξινόμηση}
Ο κώδικας της γρήγορης ταξινόμησης  παρουσιάζεται στη συνέχεια. Πρόκειται για κώδικα ο οποίος καλείται αναδρομικά σε υποακολουθίες των δεδομένων και σε κάθε κλήση επιλέγει ένα στοιχείο (pivot) και διαχωρίζει τα υπόλοιπα στοιχεία έτσι ώστε αριστερά να είναι τα στοιχεία που είναι μικρότερα του pivot και δεξιά αυτά τα οποία είναι μεγαλύτερα. 

\lstinputlisting[caption = Ο αλγόριθμος γρήγορης ταξινόμησης (quick\_sort.cpp)]{lab03/quick_sort.cpp}

\lstinputlisting[caption = sort3.cpp]{lab03/sort3.cpp}

\lstinputlisting[style=DOS]{lab03/sort3.out}

\subsection{Ταξινόμηση κατάταξης}
ο αλγόριθμος ταξινόμησης κατάταξης (rank-sort) λειτουργεί ως εξής: Για κάθε στοιχείο του δεδομένου πίνακα a που επιθυμούμε να ταξινομήσουμε υπολογίζεται μια τιμή κατάταξης (rank). Η τιμή κατάταξης ενός στοιχείου του πίνακα είναι το πλήθος των μικρότερων από αυτό στοιχείων συν το πλήθος των ίσων με αυτό στοιχείων που έχουν μικρότερο δείκτη σε σχέση με αυτό το στοιχείο (δηλαδή βρίσκονται αριστερά του).  Δηλαδή ισχύει ότι η τιμή κατάταξης ενός στοιχείου x του πίνακα είναι ίση με το άθροισμα 2 όρων: του πλήθους των μικρότερων στοιχείων του x από όλο τον πίνακα  και του πλήθους των ίσων με το x στοιχείων που έχουν μικρότερο δείκτη σε σχέση με το x. Για παράδειγμα στην ακολουθία τιμών a=[44, 21, 78, 16, 56, 21] θα πρέπει να δημιουργηθεί ένας νέος πίνακας r = [3, 1, 5, 0, 4, 2]. Έχοντας υπολογίσει τον πίνακα r θα πρέπει τα στοιχεία του a να αντιγραφούν σε ένα νέο βοηθητικό πίνακα temp έτσι ώστε κάθε τιμή που υπάρχει στον πίνακα r να λειτουργεί ως δείκτης για το που πρέπει να τοποθετηθεί το αντίστοιχο στοιχείο του a στον πίνακα temp. Τέλος θα πρέπει να αντιγραφεί ο πίνακας temp στον πίνακα a.
Στη συνέχεια παρουσιάζεται ο κώδικας του αλγορίθμου rank-sort. Παρουσιάζονται δύο υλοποιήσεις. Η πρώτη υλοποίηση (rank\_sort) αφορά τον αλγόριθμο όπως έχει περιγραφεί παραπάνω ενώ η δεύτερη (rank\_sort\_in\_place) είναι από το βιβλίο ``Δομές Δεδομένων, Αλγόριθμοι και Εφαρμογές στη C++ του Sartaj Sahnii, Εκδόσεις Τζιόλα, 2004'' στη σελίδα 63 (πρόγραμμα 2.11) και δεν απαιτεί τη χρήση του βοηθητικού πίνακα temp, συνεπώς είναι αποδοτικότερος. 

\lstinputlisting[caption = Ο αλγόριθμος ταξινόμησης κατάταξης (sort4.cpp)]{lab03/sort4.cpp}

\lstinputlisting[style=DOS]{lab03/sort4.out}

\section{Αλγόριθμοι αναζήτησης}

\subsection{Σειριακή αναζήτηση}
Η σειριακή αναζήτηση είναι ο απλούστερος αλγόριθμος αναζήτησης. Εξετάζει τα στοιχεία ένα προς ένα στη σειρά μέχρι να βρει το στοιχείο που αναζητείται. Το πλεονέκτημα του αλγορίθμου είναι ότι μπορεί να εφαρμοστεί σε μη ταξινομημένους πίνακες.

\lstinputlisting[caption = Ο αλγόριθμος σειριακής αναζήτησης (search1.cpp)]{lab03/search1.cpp}

\lstinputlisting[style=DOS]{lab03/search1.out}

\subsection{Δυαδική αναζήτηση}
Η δυαδική αναζήτηση μπορεί να εφαρμοστεί μόνο σε ταξινομημένα δεδομένα. Διαιρεί επαναληπτικά την ακολουθία σε 2 υποακολουθίες και απορρίπτει την ακολουθία στην οποία συμπεραίνει ότι δεν μπορεί να βρεθεί το στοιχείο. 

\lstinputlisting[caption = Ο αλγόριθμος δυαδικής αναζήτησης (binary\_search.cpp)]{lab03/binary_search.cpp}

\lstinputlisting[caption = search2.cpp]{lab03/search2.cpp}

\lstinputlisting[style=DOS]{lab03/search2.out}

\subsection{Αναζήτηση με παρεμβολή}
Η αναζήτηση με παρεμβολή (interpolation-search) είναι μια παραλλαγή της δυαδικής αναζήτησης και μπορεί να εφαρμοστεί μόνο σε ταξινομημένα δεδομένα. Αντί να χρησιμοποιηθεί η τιμή 50\% για να διαχωριστούν τα δεδομένα σε 2 ισομεγέθεις λίστες (όπως συμβαίνει στη δυαδική αναζήτηση) υπολογίζεται μια τιμή η οποία εκτιμάται ότι θα οδηγήσει πλησιέστερα στο στοιχείο που αναζητείται. Αν l είναι ο δείκτης του αριστερότερου στοιχείου της ακολουθίας και r o δείκτης του δεξιότερου στοιχείου της ακολουθίας τότε υπολογίζεται ο συντελεστής 
c = (key−a[l])/(a[r]−a[l]) όπου key είναι το στοιχείο προς αναζήτηση και a είναι η ακολουθία τιμών στην οποία αναζητείται το key. Η ακολουθία των δεδομένων διαχωρίζεται με βάση τον συντελεστή c σε δύο υποακολουθίες. Η διαδικασία επαναλαμβάνεται ανάλογα με τη δυαδική αναζήτηση. Στη συνέχεια παρουσιάζεται ο κώδικας της αναζήτησης με παρεμβολή.

\lstinputlisting[caption = Ο αλγόριθμος αναζήτησης με παρεμβολή (interpolation\_search.cpp)]{lab03/interpolation_search.cpp}

\lstinputlisting[caption = search3.cpp]{lab03/search3.cpp}

\lstinputlisting[style=DOS]{lab03/search3.out}

\section{Παραδείγματα}
\subsection{Παράδειγμα 1}
Γράψτε πρόγραμμα που να συγκρίνει τους χρόνους εκτέλεσης των αλγορίθμων ταξινόμησης insertion-sort, merge-sort, quick-sort καθώς και του αλγορίθμου ταξινόμησης της βιβλιοθήκης STL (συνάρτηση sort). Η σύγκριση να αφορά τυχαία δεδομένα τύπου float με τιμές στο διάστημα από -1.000 έως 1.000. Τα μεγέθη των πινάκων που θα ταξινομηθούν να είναι 5.000, 10.000, 20.0000, 40.000, 80.000, 160.0000 και 320.000 αριθμών. 

\lstinputlisting[caption = Σύγκριση χρόνου εκτέλεσης αλγορίθμων ταξινόμησης (lab03\_ex1.cpp)]{lab03/lab03_ex1.cpp}

\lstinputlisting[style=DOS]{lab03/lab03_ex1.out}


\subsection{Παράδειγμα 2}
Γράψτε πρόγραμμα που να συγκρίνει τους χρόνους εκτέλεσης των αλγορίθμων αναζήτησης binary-search, interpolation-search και του αλγορίθμου αναζήτησης της βιβλιοθήκης STL binary\_search για ταξινομημένα ακέραια δεδομένα με τιμές στο διάστημα από 0  έως 10.000.000. Η σύγκριση να εξετάζει τα ακόλουθα μεγέθη πινάκων 5.000, 10.000, 20.000, 40.000, 80.000, 160.000 και 320.000 αριθμών. Οι χρόνοι εκτέλεσης να αφορούν τους συνολικούς χρόνους που απαιτούνται έτσι ώστε να αναζητηθούν 100.000 τυχαίες τιμές με καθένα από τους αλγορίθμους. 


\lstinputlisting[caption = Σύγκριση χρόνου εκτέλεσης αλγορίθμων ταξινόμησης (lab03\_ex2.cpp)]{lab03/lab03_ex2.cpp}

\lstinputlisting[style=DOS]{lab03/lab03_ex2.out}

\section{Ασκήσεις}
\begin{enumerate}
\item Ο αλγόριθμος bogosort αναδιατάσσει τυχαία τις τιμές ενός πίνακα μέχρι να προκύψει μια ταξινομημένη διάταξη. Γράψτε ένα πρόγραμμα που να υλοποιεί τον αλγόριθμο bogosort για την ταξινόμηση ενός πίνακα ακεραίων τιμών.  

\end{enumerate}







% Εργαστήριο 4
\chapter{Γραμμικές λίστες, λίστες της STL}
\section{Εισαγωγή}
Οι γραμμικές λίστες είναι δομές δεδομένων που επιτρέπουν την αποθήκευση και την προσπέλαση στοιχείων έτσι ώστε τα στοιχεία να βρίσκονται σε μια σειρά με σαφώς ορισμένη την έννοια της θέσης καθώς και το ποιο στοιχείο προηγείται και ποιο έπεται καθενός. Σε χαμηλού επιπέδου γλώσσες προγραμματισμού όπως η C η υλοποίηση γραμμικών λιστών είναι ευθύνη του προγραμματιστή. Από την άλλη μεριά, γλώσσες υψηλού επιπέδου όπως η C++, η Java, η Python κ.α. προσφέρουν έτοιμες υλοποιήσεις γραμμικών λιστών. Ωστόσο, η γνώση υλοποίησης των συγκεκριμένων δομών (όπως και άλλων) αποτελεί βασική ικανότητα η οποία αποκτά ιδιαίτερη χρησιμότητα όταν ζητούνται εξειδικευμένες υλοποιήσεις. Στο συγκεκριμένο εργαστήριο θα παρουσιαστούν δύο πιθανές υλοποιήσεις γραμμικών λιστών (στατικής λίστας και απλά συνδεδεμένης λίστας) καθώς και οι ενσωματωμένες δυνατότητες της C++ μέσω containers της STL όπως το vector και το list. Ο κώδικας όλων των παραδειγμάτων βρίσκεται στο \href{https://github.com/chgogos/ceteiep_dsa}{https://github.com/chgogos/ceteiep\_dsa}.

\section{Γραμμικές λίστες}
Υπάρχουν δύο βασικοί τρόποι αναπαράστασης γραμμικών λιστών, η στατική αναπαράσταση η οποία γίνεται με τη χρήση πινάκων και η αναπαράσταση με συνδεδεμένη λίστα η οποία γίνεται με τη χρήση δεικτών. 

\subsection{Στατικές γραμμικές λίστες}
Στη στατική γραμμική λίστα τα δεδομένα αποθηκεύονται σε ένα πίνακα. Κάθε στοιχείο της στατικής λίστας μπορεί να προσπελαστεί με βάση τη θέση του στον ίδιο σταθερό χρόνο με όλα τα άλλα στοιχεία άσχετα με τη θέση στην οποία βρίσκεται (τυχαία προσπέλαση). Ο κώδικας υλοποίησης μιας στατικής λίστας με μέγιστη χωρητικότητα 50.000 στοιχείων παρουσιάζεται στη συνέχεια.

\lstinputlisting[caption = Υλοποίηση στατικής γραμμικής λίστας (static\_list.cpp),label=lst:static_list.cpp]{lab04/static_list.cpp}

\lstinputlisting[caption = Παράδειγμα με στατική γραμμική λίστα (list1.cpp),label=lst:list1.cpp]{lab04/list1.cpp}

Στους κώδικες που προηγήθηκαν καθώς και σε επόμενους γίνεται χρήση εξαιρέσεων (exceptions) για να σηματοδοτηθούν γεγονότα τα οποία αφορούν έκτακτες καταστάσεις που το πρόγραμμα θα πρέπει να διαχειρίζεται. Για παράδειγμα, όταν επιχειρηθεί η προσπέλαση ενός στοιχείου σε μια θέση εκτός των ορίων της λίστας (π.χ. ενέργεια 5 στον κώδικα \ref{lst:list1.cpp}) τότε γίνεται throw ένα exception out\_of\_range το οποίο θα πρέπει να συλληφθεί (να γίνει catch) από τον κώδικα που καλεί τη συνάρτηση που προκάλεσε το throw exception. Περισσότερες πληροφορίες για τα exceptions και τον χειρισμό τους μπορούν να αναζητηθούν στην αναφορά \cite{cppexceptions}.
\lstinputlisting[style=DOS]{lab04/list1.out}

Οι στατικές γραμμικές λίστες έχουν τα ακόλουθα πλεονεκτήματα:
\begin{itemize}[nolistsep]
\item Εύκολη υλοποίηση. 
\item Σταθερός χρόνος $O(1)$ εντοπισμού στοιχείου με βάση τη θέση του.
\item Γραμμικός χρόνος $O(n)$ για αναζήτηση ενός στοιχείου ή λογαριθμικός χρόνος $O(log(n))$ αν τα στοιχεία είναι ταξινομημένα.
\end{itemize}

Ωστόσο, οι στατικές γραμμικές λίστες έχουν και μειονεκτήματα τα οποία παρατίθενται στη συνέχεια:
\begin{itemize}[nolistsep]
\item Δέσμευση μεγάλου τμήματος μνήμης ακόμη και όταν η λίστα είναι άδεια ή περιέχει λίγα στοιχεία. 
\item Επιβολή άνω ορίου στα δεδομένα τα οποία μπορεί να δεχθεί (ο περιορισμός αυτός μπορεί να ξεπεραστεί με συνθετότερη υλοποίηση που αυξομειώνει το μέγεθος του πίνακα υποδοχής όταν αυτό απαιτείται).
\item Γραμμικός χρόνος $O(n)$ για εισαγωγή και διαγραφή στοιχείων του πίνακα.
\end{itemize}


\subsection{Συνδεδεμένες γραμμικές λίστες}
Η συνδεδεμένη γραμμική λίστα αποτελείται από μηδέν ή περισσότερους κόμβους. Κάθε κόμβος περιέχει δεδομένα και έναν ή περισσότερους δείκτες σε άλλους κόμβους της συνδεδεμένης λίστας. Συχνά χρησιμοποιείται ένας πρόσθετος κόμβος με όνομα head (κόμβος κεφαλής) που δείχνει στο πρώτο στοιχείο της λίστας και μπορεί να περιέχει επιπλέον πληροφορίες όπως το μήκος της. Στη συνέχεια παρουσιάζεται ο κώδικας που υλοποιεί μια απλά συνδεδεμένη λίστα.

\lstinputlisting[caption = Υλοποίηση συνδεδεμένης γραμμικής λίστας (linked\_list.cpp),label=lst:linked_list.cpp]{lab04/linked_list.cpp}

\lstinputlisting[caption = Παράδειγμα με συνδεδεμένη γραμμική λίστα (list2.cpp)]{lab04/list2.cpp}

\lstinputlisting[style=DOS]{lab04/list2.out}


Οι συνδεδεμένες γραμμικές λίστες έχουν τα ακόλουθα πλεονεκτήματα:
\begin{itemize}[nolistsep]
\item Καλή χρήση του αποθηκευτικού χώρου (αν και απαιτείται περισσότερος χώρος για την αποθήκευση κάθε κόμβου λόγω των δεικτών). 
\item Σταθερός χρόνος $O(1)$ για την εισαγωγή και διαγραφή στοιχείων.
\end{itemize}
Από την άλλη μεριά τα μειονεκτήματα των συνδεδεμένων λιστών είναι τα ακόλουθα:
\begin{itemize}[nolistsep]
\item Συνθετότερη υλοποίηση.
\item Δεν επιτρέπουν την απευθείας μετάβαση σε κάποιο στοιχείο με βάση τη θέση του.
\end{itemize}

Οι αναφορές \cite{csstanford103} και \cite{csstanford105} παρέχουν χρήσιμες πληροφορίες και ασκήσεις σχετικά με τις συνδεδεμένες λίστες και το ρόλο των δεικτών στην υλοποίησή τους.

\subsection{Γραμμικές λίστες της STL}
Τα containers της STL που μπορούν να λειτουργήσουν ως διατεταγμένες συλλογές (ordered collections) είναι τα ακόλουθα: vector, deque, arrays, list, forward\_list και bitset. 

\subsubsection{Vectors}
Τα vectors αλλάζουν αυτόματα μέγεθος καθώς προστίθενται ή αφαιρούνται στοιχεία σε αυτά. Τα δεδομένα τους τοποθετούνται σε συνεχόμενες θέσεις μνήμης. Περισσότερες πληροφορίες για τα vectors μπορούν να βρεθούν στις αναφορές \cite{g4gvector} και \cite{codecogsvector}.

\subsubsection{Deques}
Τα deques (double ended queues = ουρές με δύο άκρα) είναι παρόμοια με τα vectors αλλά μπορούν να προστεθούν ή να διαγραφούν στοιχεία τόσο από την αρχή όσο και από το τέλος τους. Περισσότερες πληροφορίες για τα deques μπορούν να βρεθούν στην αναφορά \cite{g4gdeque}.

\subsubsection{Arrays}
Τα arrays εισήχθησαν στη C++11 με στόχο να αντικαταστήσουν τους απλούς πίνακες της C. Κατά τη δήλωση ενός array προσδιορίζεται και το μέγεθός του. Περισσότερες πληροφορίες για τα arrays μπορούν να βρεθούν στην αναφορά \cite{g4garray}.

\subsubsection{Lists}
Οι lists είναι διπλά συνδεδεμένες λίστες. Δηλαδή κάθε κόμβος της λίστας διαθέτει έναν δείκτη προς το επόμενο και έναν δείκτη προς το προηγούμενο στοιχείο στη λίστα. Περισσότερες πληροφορίες για τις lists μπορούν να βρεθούν στην αναφορά \cite{g4glist}.

\subsubsection{Forward Lists}
Οι forward lists είναι απλά συνδεδεμένες λίστες με κάθε κόμβο να διαθέτει έναν δείκτη προς το επόμενο στοιχείο της λίστας. Περισσότερες πληροφορίες για τις forward lists μπορούν να βρεθούν στις αναφορές \cite{g4gforwardlist1} και \cite{g4gforwardlist2}.

\subsubsection{Bitset}
Τα bitsets είναι πίνακες με λογικές τιμές τις οποίες αποθηκεύουν με αποδοτικό τρόπο καθώς για κάθε λογική τιμή απαιτείται μόνο 1 bit. Το μέγεθος ενός bitset πρέπει να είναι γνωστό κατά τη μεταγλώττιση. Μια ιδιαιτερότητά του είναι ότι οι δείκτες θέσης που χρησιμοποιούνται για την αναφορά στα στοιχεία του ξεκινούν την αρίθμησή τους με το μηδέν από δεξιά και αυξάνονται προς τα αριστερά. Για παράδειγμα ένα bitset με τιμές 101011 έχει την τιμή 1 στις θέσεις 0,1,3,5 και 0 στις θέσεις 2 και 4. Περισσότερες πληροφορίες για τα bitsets μπορούν να βρεθούν στις αναφορές \cite{g4gbitset1} και \cite{g4gbitset2}.


\section{Παραδείγματα}
\subsection{Παράδειγμα 1}
Γράψτε ένα πρόγραμμα που να ελέγχεται από το ακόλουθο μενού και να πραγματοποιεί τις λειτουργίες που περιγράφονται σε μια απλά συνδεδεμένη λίστα με ακεραίους.
\begin{enumerate}[nolistsep]
\item Εμφάνιση στοιχείων λίστας. (Show list)
\item Εισαγωγή στοιχείου στο πίσω άκρο της λίστας. (Insert item (back))
\item Εισαγωγή στοιχείου σε συγκεκριμένη θέση. (Insert item (at position)) 
\item Διαγραφή στοιχείου σε συγκεκριμένη θέση. (Delete item (from position))
\item Διαγραφή όλων των στοιχείων που έχουν την τιμή. (Delete all items having value)
\item Έξοδος. (Exit)
\end{enumerate}

\lstinputlisting[caption = Έλεγχος συνδεδεμένης λίστας ακεραίων μέσω μενού (lab04\_ex1.cpp)]{lab04/lab04_ex1.cpp}

\lstinputlisting[style=DOS]{lab04/lab04_ex1.out}


\subsection{Παράδειγμα 2}
Έστω μια τράπεζα που διατηρεί για κάθε πελάτη της το ονοματεπώνυμό του και το υπόλοιπο του λογαριασμού του. Για τις ανάγκες του παραδείγματος θα δημιουργηθούν τυχαίοι πελάτες ως εξής: το όνομα κάθε πελάτη θα αποτελείται από 10 γράμματα που θα επιλέγονται με τυχαίο τρόπο από τα γράμματα της αγγλικής αλφαβήτου και το δε υπόλοιπο κάθε πελάτη θα είναι ένας τυχαίος ακέραιος αριθμός από το 0 μέχρι το 5.000. Το πρόγραμμα θα πραγματοποιεί τις ακόλουθες λειτουργίες: 
\begin{enumerate}[noitemsep,label=\Alph*]
\item Θα δημιουργεί μια λίστα με 40.000 τυχαίους πελάτες.
\item Θα υπολογίζει το άθροισμα των υπολοίπων από όλους τους πελάτες που το όνομά τους ξεκινά με το χαρακτήρα Α.
\item Θα προσθέτει για κάθε πελάτη που το όνομά του ξεκινά με το χαρακτήρα G στην αμέσως επόμενη θέση έναν πελάτη με όνομα το αντίστροφο όνομα του πελάτη και το ίδιο υπόλοιπο λογαριασμού.
\item Θα διαγράφει όλους τους πελάτες που το όνομά τους ξεκινά με το χαρακτήρα Β.
\end{enumerate}
Τα δεδομένα θα αποθηκεύονται σε μια συνδεδεμένη λίστα πραγματοποιώντας χρήση του κώδικα \ref{lst:linked_list.cpp} καθώς και άλλων συναρτήσεων που επιτρέπουν την αποδοτικότερη υλοποίηση των παραπάνω ερωτημάτων.

\lstinputlisting[caption = Λίστα πελατών για το ίδιο πρόβλημα (lab04\_ex2.cpp)]{lab04/lab04_ex2.cpp}

\lstinputlisting[style=DOS]{lab04/lab04_ex2.out}

Αν στη θέση της συνδεδεμένης λίστας του κώδικα \ref{lst:linked_list.cpp} χρησιμοποιηθεί η στατική λίστα του κώδικα \ref{lst:static_list.cpp} ή ένα vector ή ένα list της STL τα αποτελέσματα θα είναι τα ίδια αλλά η απόδοση στα επιμέρους ερωτήματα του παραδείγματος θα διαφέρει όπως φαίνεται στον πίνακα \ref{tbl:lists}. Ο κώδικας που παράγει τα αποτελέσματα βρίσκεται στο αρχείο lab04/lab04\_ex2\_x4.cpp.

\begin{table}[!htb]
\centering
\begin{tabular}{|l|c|c|c|c|}
\hline
            & Ερώτημα A     & Ερώτημα B     & Ερώτημα Γ     & Ερώτημα Δ     \\ \hline
Συνδεδεμένη  λίστα & 0.030 & 0.001 & 0.002 & 0.001 \\ \hline
Στατική λίστα  & 0.034 & 0.003 & 0.642 & 0.671 \\ \hline
std::vector     & 0.033 & 0.002 & 0.543 & 0.519 \\ \hline
std::list      & 0.033 & 0.002 & 0.002 & 0.001 \\ \hline
\end{tabular}
\caption{Χρόνοι εκτέλεσης σε δευτερόλεπτα των ερωτημάτων του παραδείγματος 2 ανάλογα με τον τρόπο υλοποίησης της λίστας}
\label{tbl:lists}
\end{table}



\section{Ασκήσεις}
\begin{enumerate}
\item Έστω η συνδεδεμένη λίστα που παρουσιάστηκε στον κώδικα \ref{lst:linked_list.cpp}. Προσθέστε μια συνάρτηση έτσι ώστε για  μια λίστα ταξινομημένων στοιχείων από το μικρότερο προς το μεγαλύτερο, να προσθέτει ένα ακόμα στοιχείο στην κατάλληλη θέση έτσι ώστε η λίστα να παραμένει ταξινομημένη.
\item Έστω η συνδεδεμένη λίστα που παρουσιάστηκε στον κώδικα \ref{lst:linked_list.cpp}. Προσθέστε μια συνάρτηση που να αντιστρέφει τη λίστα.
\item Υλοποιήστε τη στατική λίστα (κώδικας \ref{lst:static_list.cpp}) και τη συνδεδεμένη λίστα (κώδικας \ref{lst:linked_list.cpp}) με κλάσεις. Τροποποιήστε το παράδειγμα 1 έτσι ώστε να δίνεται επιλογή στο χρήστη να χρησιμοποιήσει είτε τη στατική είτε τη συνδεδεμένη λίστα προκειμένου να εκτελέσει τις ίδιες λειτουργίες πάνω σε μια λίστα. 
\item Υλοποιήστε μια κυκλικά συνδεδεμένη λίστα. Η κυκλική λίστα είναι μια απλά συνδεδεμένη λίστα στην οποία το τελευταίο στοιχείο της λίστας δείχνει στο πρώτο στοιχείο της λίστας. Η υλοποίηση θα πρέπει να συμπεριλαμβάνει και δύο δείκτες, έναν που να δείχνει στο πρώτο στοιχείο της λίστας και έναν που να δείχνει στο τελευταίο στοιχείο της λίστας. Προσθέστε τις απαιτούμενες λειτουργίες έτσι ώστε η λίστα να παρέχει τις ακόλουθες λειτουργίες: εμφάνιση λίστας, εισαγωγή στοιχείου, διαγραφή στοιχείου, εμφάνιση πλήθους στοιχείων, εύρεση στοιχείου. Γράψτε πρόγραμμα που να δοκιμάζει τις λειτουργίες της λίστας.
\end{enumerate}

\begin{thebibliography}{9}
\bibitem{cppexceptions}
C++ Tutorial-exceptions-2017 by K. Hong, \href{http://www.bogotobogo.com/cplusplus/exceptions.php}{http://www.bogotobogo.com/cplusplus/exceptions.php}.
\bibitem{csstanford103}
Linked List Basics by N. Parlante, \href{http://cslibrary.stanford.edu/103/}{http://cslibrary.stanford.edu/103/}.
\bibitem{csstanford105}
Linked List Problems by N. Parlante, \href{http://cslibrary.stanford.edu/105/}{http://cslibrary.stanford.edu/105/}.
\bibitem{g4gvector}
Geeks for Geeks, Vector in C++ STL, \href{http://www.geeksforgeeks.org/vector-in-cpp-stl/}{http://www.geeksforgeeks.org/vector-in-cpp-stl/}.
\bibitem{codecogsvector}
Codecogs, Vector, a random access dynamic container, \href{http://www.codecogs.com/library/computing/stl/containers/sequences/vector.php}{http://www.codecogs.com/library/computing}.
\bibitem{g4gdeque}
Geeks for Geeks, Deque in C++ STL, \href{http://www.geeksforgeeks.org/deque-cpp-stl/}{http://www.geeksforgeeks.org/deque-cpp-stl/}.
\bibitem{g4garray}
Geeks for Geeks, Array class in C++ STL \href{http://www.geeksforgeeks.org/array-class-c/}{http://www.geeksforgeeks.org/array-class-c/}.
\bibitem{g4glist}
Geeks for Geeks, List in C++ STL  \href{http://www.geeksforgeeks.org/list-cpp-stl/}{http://www.geeksforgeeks.org/list-cpp-stl/}
\bibitem{g4gforwardlist1}
Geeks for Geeks, Forward List in C++ (Set 1) \href{http://www.geeksforgeeks.org/forward-list-c-set-1-introduction-important-functions/}{http://www.geeksforgeeks.org/forward-list-c-set-1-introduction-important-functions/}
\bibitem{g4gforwardlist2}
Geeks for Geeks, Forward List in C++ (Set 2) \href{http://www.geeksforgeeks.org/forward-list-c-set-2-manipulating-functions/}{http://www.geeksforgeeks.org/forward-list-c-set-2-manipulating-functions/}
\bibitem{g4gbitset1}
Geeks for Geeks, C++ bitset and its application, \href{http://www.geeksforgeeks.org/c-bitset-and-its-application/}{http://www.geeksforgeeks.org/c-bitset-and-its-application/}
\bibitem{g4gbitset2}
Geeks for Geeks, C++ bitset interesting facts, \href{http://www.geeksforgeeks.org/c-bitset-interesting-facts/}{http://www.geeksforgeeks.org/c-bitset-interesting-facts/}
\end{thebibliography}



% Εργαστήριο 5
\chapter{Στοίβες και ουρές, οι δομές στοίβα και ουρά στην STL}
\chaptermark{Στοίβες, ουρές}
\section{Εισαγωγή}
Ο κώδικας όλων των παραδειγμάτων βρίσκεται στο \href{https://github.com/chgogos/ceteiep_dsa}{https://github.com/chgogos/ceteiep\_dsa}.

\section{Στοίβα}
Η στοίβα (stack) είναι μια ειδική περίπτωση γραμμικής λίστας στην οποία οι εισαγωγές και οι διαγραφές επιτρέπονται μόνο από το ένα άκρο. Συνήθως αυτό το άκρο λέγεται κορυφή (top). Πρόκειται για μια δομή στην οποία οι εισαγωγές και οι εξαγωγές γίνονται σύμφωνα με το μοντέλο τελευταίο μέσα πρώτο έξω (LIFO=Last In First Out).

Στη συνέχεια παρουσιάζεται μια υλοποίηση στοίβας που χρησιμοποιεί συνδεδεμένη λίστα της STL για την αποθήκευση των στοιχείων της.

\section{Ουρά}
Η ουρά (queue) είναι μια ειδική περίπτωση γραμμικής λίστας στην οποία επιτρέπονται εισαγωγές στο πίσω άκρο της και εξαγωγές από το εμπρός άκρο της μόνο. Τα δύο αυτά άκρα συνήθως αναφέρονται ως
πίσω (rear) και εμπρός (front) αντίστοιχα. Η ουρά είναι μια δομή στην οποία οι εισαγωγές και οι εξαγωγές γίνονται σύμφωνα με το μοντέλο πρώτο μέσα πρώτο έξω (FIFO=First In First Out).

Στη συνέχεια παρουσιάζεται μια υλοποίηση ουράς που χρησιμοποιεί συνδεδεμένη λίστα της STL για την αποθήκευση των στοιχείων της.

\section{Παραδείγματα}

\section{Ασκήσεις}
\begin{enumerate}
\item a
\item b
\item c
\item d
\end{enumerate}

\begin{thebibliography}{9}

\end{thebibliography}



% Εργαστήριο 6
\chapter{Σωροί μεγίστων και σωροί ελαχίστων, η ταξινόμηση heapsort, ουρές προτεραιότητας στην STL}
\chaptermark{Σωροί}
\section{Εισαγωγή}
Ο κώδικας όλων των παραδειγμάτων βρίσκεται στο \href{https://github.com/chgogos/ceteiep_dsa}{https://github.com/chgogos/ceteiep\_dsa}.

\section{Σωροί}
Ο σωρός είναι μια μερικά ταξινομημένη δομή δεδομένων. Υπάρχουν δύο ειδών σωροί, ο σωρός μεγίστων (Max-Heap) και ο σωρός ελαχίστων (Min-Heap). Οι ιδιότητες των σωρών που θα περιγραφούν στη συνέχεια αφορούν τους σωρούς μεγίστων αλλά αντίστοιχες ιδιότητες ισχύουν και για τους σωρούς ελαχίστων. Ειδικότερα, ένας σωρός μεγίστων υποστηρίζει ταχύτατα τις ακόλουθες λειτουργίες:
\begin{itemize}[noitemsep]
\item Εύρεση του στοιχείου με τη μεγαλύτερη τιμή κλειδιού.
\item Διαγραφή του στοιχείου με τη μεγαλύτερη τιμή κλειδιού.
\item Εισαγωγή νέου κλειδιού στη δομή.
\end{itemize}

Ένας σωρός μπορεί να θεωρηθεί ως ένα δυαδικό δένδρο για το οποίο ισχύουν οι ακόλουθοι δύο περιορισμοί:
\begin{itemize}[noitemsep]
\item	Πληρότητα: το δυαδικό δένδρο είναι συμπληρωμένο, δηλαδή όλα τα επίπεδά του είναι πλήρως συμπληρωμένα εκτός πιθανά από το τελευταίο επίπεδο στο οποίο μπορούν να λείπουν μόνο κάποια από τα δεξιότερα φύλλα.
\item	Κυριαρχία γονέα: το κλειδί σε κάθε κορυφή είναι μεγαλύτερο ή ίσο από τα κλειδιά των παιδιών (σε Max-Heap).
\end{itemize}

Στο σχήμα \ref{fig:heap1} παρουσιάζεται ένα παράδειγμα σωρού μεγίστων στη δενδρική του μορφή.

\begin{figure}[ht]
\centering
\includegraphics[width=100mm]{heap1.png}
\caption{Σωρός μεγίστων στη δενδρική του απεικόνιση}
\label{fig:heap1}
\end{figure}

Ένας σωρός μπορεί να υλοποιηθεί με ένα πίνακα καταγράφοντας στον πίνακα στη σειρά τα στοιχεία του δυαδικού δένδρου από αριστερά προς τα δεξιά και από πάνω προς τα κάτω (σχήμα \ref{fig:heap2}). Μερικές σημαντικές ιδιότητες οι οποίες προκύπτουν εφόσον τηρηθεί ο παραπάνω τρόπος αντιστοίχισης των στοιχείων του δένδρου στα στοιχεία του  πίνακα είναι οι ακόλουθες:
\begin{itemize}[noitemsep]
\item	Στον πίνακα τα κελιά γονείς βρίσκονται στις πρώτες $\lfloor{\frac{n}{2}}\rfloor$ θέσεις ενώ τα φύλλα καταλαμβάνουν τις υπόλοιπες θέσεις. 
\item	Στον πίνακα τα παιδιά για κάθε κλειδί στις θέσεις $i$ από 1 μέχρι και $\lfloor{\frac{n}{2}}\rfloor$ βρίσκονται στις θέσεις $2*i$ και $2*i + 1$.
\item	Στον πίνακα ο γονέας για κάθε κλειδί στις θέσεις $i$ από 2 μέχρι και $n$ βρίσκεται στη θέση $\lfloor{\frac{i}{2}}\rfloor$.
\end{itemize}

\begin{figure}[ht]
\centering
\includegraphics[width=100mm]{heap2.png}
\caption{Αναπαράσταση ενός σωρού μεγίστων ως πίνακα}
\label{fig:heap2}
\end{figure}

\section{Υλοποίηση ενός σωρού}
Στη συνέχεια παρουσιάζεται η υλοποίηση ενός σωρού μεγίστων που περιέχει ακέραιες τιμές-κλειδιά.

\lstinputlisting[caption = Σωρός μεγίστων ακεραίων (max\_heap.cpp)]{lab06/max_heap.cpp}

Ο ακόλουθος κώδικας χρησιμοποιεί τη συνάρτηση heap\_bottom\_up και μέσω αυτής τη συνάρτηση heapify προκειμένου να μετασχηματίσει έναν πίνακα ακεραίων σε σωρό μεγίστων.
\lstinputlisting[caption = Δημιουργία σωρού από πίνακα με heapify (heap1.cpp)]{lab06/heap1.cpp}

\lstinputlisting[style=DOS]{lab06/heap1.out}

Ο ακόλουθος κώδικας δημιουργεί σταδιακά έναν σωρό εισάγοντας δέκα τιμές με τη συνάρτηση insert\_key. Στη συνέχεια πραγματοποιούνται εξαγωγές τιμών με τη συνάρτηση maximum\_key\_deletion μέχρι ο σωρός να αδειάσει.

\lstinputlisting[caption = Δημιουργία σωρού με εισαγωγές τιμών και εν συνεχεία άδειασμα του σωρού με διαδοχικές διαγραφές της μέγιστης τιμής (heap2.cpp)]{lab06/heap2.cpp}

\lstinputlisting[style=DOS]{lab06/heap2.out}


\section{Ταξινόμηση Heapsort}
Ο αλγόριθμος Heapsort προτάθηκε από τον J.W.J.Williams το 1964 και αποτελείται από 2 στάδια:
\begin{itemize}[noitemsep]
\item Δημιουργία σωρού με τα n στοιχεία ενός πίνακα τα στοιχεία του οποίου ζητείται να ταξινομηθούν. 
\item Εφαρμογή της διαγραφής της ρίζας n -1 φορές.
\end{itemize}
Το αποτέλεσμα είναι ότι τα στοιχεία αφαιρούνται από το σωρό σε φθίνουσα σειρά. Καθώς κατά την αφαίρεσή του κάθε στοιχείου, αυτό τοποθετείται στο τέλος του σωρού, τελικά ο σωρός περιέχει τα αρχικά δεδομένα σε αύξουσα σειρά. Στη συνέχεια παρουσιάζεται η υλοποίηση του αλγορίθμου HeapSort. Επιπλέον ο κώδικας ταξινομεί πίνακες μεγέθους 10.000, 20.000, 40.000 80.000 και 100.000 που περιέχουν τυχαίες ακέραιες τιμές και πραγματοποιείται σύγκριση με τους χρόνους εκτέλεσης που επιτυγχάνει η std::sort.

\lstinputlisting[caption = Ο αλγόριθμος heapsort (heapsort.cpp)]{lab06/heapsort.cpp}

\lstinputlisting[style=DOS]{lab06/heapsort.out}


\section{Η δομή priority\_queue της STL}
Η STL της C++ περιέχει υλοποίηση της δομής std::priority\_queue (ουρά προτεραιότητας) η οποία είναι ένας σωρός μεγίστων. Κάθε στοιχείο που εισέρχεται  στην ουρά προτεραιότητας έχει μια προτεραιότητα που συνδέεται με αυτό και το στοιχείο με τη μεγαλύτερη προτεραιότητα βρίσκεται πάντα στην αρχή της ουράς. Οι κυριότερες λειτουργίες που υποστηρίζονται από την std::priority\_queue είναι οι ακόλουθες:
\begin{itemize}[noitemsep]
\item push: εισαγωγή ενός στοιχείου στη δομή
\item top: επιστροφή χωρίς εξαγωγή του στοιχείου με τη μεγαλύτερη προτεραιότητα
\item pop: απώθηση του στοιχείου με τη μεγαλύτερη προτεραιότητα
\item size: πλήθος των στοιχείων που υπάρχουν στη δομή
\item empty: επιστρέφει true αν η δομή είναι άδεια αλλιώς επιστρέφει false
\end{itemize}
Ένα παράδειγμα χρήσης της std::priority\_queue ως σωρού μεγίστων αλλά και ως σωρού ελαχίστων παρουσιάζεται στη συνέχεια.

\lstinputlisting[caption = Παράδειγμα με priority\_queue της STL (stl\_priority\_queue.cpp)]{lab06/stl_priority_queue.cpp}

\lstinputlisting[style=DOS]{lab06/stl_priority_queue.out}


\section{Παραδείγματα}
\subsection{Παράδειγμα 1}
Διάμεσος ενός δείγματος Ν παρατηρήσεων οι οποίες έχουν διαταχθεί σε αύξουσα σειρά ορίζεται ως η μεσαία παρατήρηση, όταν το Ν είναι περιττός αριθμός, ή ο μέσος όρος (ημιάθροισμα) των δύο μεσαίων παρατηρήσεων όταν το Ν είναι άρτιος αριθμός. 
Έστω ότι για διάφορες τιμές που παράγονται με κάποιον τρόπο ζητείται ο υπολογισμός της διάμεσης  τιμής καθώς παράγεται κάθε νέα τιμή και για όλες τις τιμές που έχουν προηγηθεί μαζί με την τρέχουσα τιμή όπως φαίνεται στο επόμενο παράδειγμα:\\ 
5  $\Rightarrow$  διάμεσος 5\\
5, 7 $\Rightarrow$ διάμεσος 6\\
5, 7, 13 $\Rightarrow$ διάμεσος 7\\
5, 7, 13, 12 $\Rightarrow$ 5, 7, 12, 13 $\Rightarrow$ διάμεσος 9.5\\
5, 7, 13, 12, 2 $\Rightarrow$ 2, 5, 7, 12, 13 $\Rightarrow$ διάμεσος 7

\lstinputlisting[caption = Υπολογισμός διαμέσου σε μια ροή τιμών (lab06\_ex1.cpp)]{lab06/lab06_ex1.cpp}

\lstinputlisting[style=DOS]{lab06/lab06_ex1.out}


\subsection{Παράδειγμα 2}

\section{Ασκήσεις}
\begin{enumerate}
\item a
\item b
\item Να υλοποιηθεί ο σωρός μεγίστων που παρουσιάστηκε στον κώδικα Χ ως κλάση. 
\item d
\end{enumerate}

\begin{thebibliography}{9}
\bibitem{cppexceptions}
Geeks for Geeks, Priority Queue in C++ Standard Template Library (STL), \href{http://www.geeksforgeeks.org/priority-queue-in-cpp-stl/}{http://www.geeksforgeeks.org/priority-queue-in-cpp-stl/}
\end{thebibliography}



% Εργαστήριο 7
\chapter{Κατακερματισμός, δομές κατακερματισμού στην STL}
\chaptermark{Κατακερματισμός}
\section{Εισαγωγή}

\section{Τι είναι ο κατακερματισμός;}
Ο κατακερματισμός (hashing) είναι μια μέθοδος που επιτυγχάνει ταχύτατη αποθήκευση και αναζήτηση δεδομένων. Σε ένα σύστημα κατακερματισμού τα δεδομένα αποθηκεύονται σε έναν πίνακα που ονομάζεται πίνακας κατακερματισμού (hash table). Εφαρμόζοντας στο κλειδί κάθε εγγραφής που πρόκειται να αποθηκευτεί ή να αναζητηθεί τη συνάρτηση κατακερματισμού (hash function) προσδιορίζεται μονοσήμαντα η θέση του πίνακα στην οποία τοποθετούνται τα δεδομένα της εγγραφής.
Μια καλή συνάρτηση κατακερματισμού θα πρέπει να κατανέμει τα κλειδιά στα κελιά του πίνακα κατακερματισμού όσο πιο ομοιόμορφα γίνεται και να είναι εύκολο να υπολογιστεί.

\begin{figure}[h]
\centering
\includegraphics{HashTable.png}
\caption{}
\label{fig:hashtable1}
\end{figure}

Είναι επιθυμητό το παραγόμενο αποτέλεσμα από τη συνάρτηση κατακερματισμού να εξαρτάται από το κλειδί στο σύνολό του.

Οι πίνακες κατακερματισμού είναι ιδιαίτερα κατάλληλοι για εφαρμογές στις οποίες πραγματοποιούνται συχνές αναζητήσεις εγγραφών με δεδομένες τιμές κλειδιών. Ωστόσο, οι πίνακες κατακερματισμού έχουν και μειονεκτήματα καθώς είναι δύσκολο να επεκταθούν από τη στιγμή που έχουν δημιουργηθεί και μετά. Επίσης, η απόδοσή των πινάκων κατακερματισμού υποβαθμίζεται καθώς οι θέσεις τους γεμίζουν με στοιχεία. Συνεπώς, εφόσον ο προγραμματιστής προχωρήσει στη δική του υλοποίηση ενός πίνακα κατακερματισμού είτε θα πρέπει να γνωρίζει εκ των προτέρων το πλήθος των στοιχείων που πρόκειται να αποθηκευτούν είτε όταν αυτό απαιτηθεί να υπάρχει πρόβλεψη έτσι ώστε τα δεδομένα να μεταφέρονται σε μεγαλύτερο πίνακα κατακερματισμού.

Στις περισσότερες εφαρμογές υπάρχουν πολύ περισσότερα πιθανά κλειδιά εγγραφών από ότι θέσεις στο πίνακα κατακερματισμού. Αν για δύο ή περισσότερα κλειδιά η εφαρμογή της συνάρτησης κατακερματισμού δίνει το ίδιο αποτέλεσμα τότε λέμε ότι συμβαίνει σύγκρουση (collision) η οποία θα πρέπει να διευθετηθεί με κάποιο τρόπο. Ειδικότερα, η εύρεση μιας εγγραφής με κλειδί k είναι μια διαδικασία δύο βημάτων:
\begin{itemize}[noitemsep]
\item Εφαρμογή της συνάρτησης κατακερματισμού στο κλειδί της εγγραφής.
\item Ξεκινώντας από την θέση που υποδεικνύει η συνάρτηση κατακερματισμού στον πίνακα κατακερματισμού, εντοπισμός της εγγραφής που περιέχει το ζητούμενο κλειδί (ενδεχόμενα θα χρειαστεί να εφαρμοστεί κάποιος μηχανισμός διευθέτησης συγκρούσεων). 
\end{itemize}

\subsection{Ανοικτή διευθυνσιοδότηση}

\subsection{Κατακερματισμός με αλυσίδες}

\section{Παραδείγματα}

\subsection{Παράδειγμα 1 - υλοποίηση πίνακα κατακερματισμού για γρήγορη αποθήκευση και αναζήτηση εγγραφών}
Έστω μια επιχείρηση η οποία επιθυμεί να αποθηκεύσει τα στοιχεία των υπαλλήλων της (όνομα, διεύθυνση) σε μια δομή έτσι ώστε με βάση το όνομα του υπαλλήλου να επιτυγχάνει τη γρήγορη ανάκληση των υπόλοιπων στοιχείων των υπαλλήλων. Στη συνέχεια παρουσιάζεται η υλοποίηση ενός πίνακα κατακερματισμού στον οποίο κλειδί θεωρείται το όνομα του υπαλλήλου και η επίλυση των συγκρούσεων πραγματοποιείται με ανοικτή διευθυνσιοδότηση (open addressing) και γραμμική αναζήτηση (linear probing). Ο πίνακας κατακερματισμού μπορεί να δεχθεί το πολύ 10.000 εγγραφές υπαλλήλων. Στο παράδειγμα χρονομετρείται η εκτέλεση για 2.000, 3.000 και 8.000 υπαλλήλους. Παρατηρείται ότι λόγω των συγκρούσεων καθώς ο συντελεστής φόρτωσης του πίνακα κατακερματισμού αυξάνεται η απόδοση της δομής υποβαθμίζεται.

\subsection{Παράδειγμα 2 - Γρήγορη αποθήκευση και αναζήτηση εγγραφών με τη χρήση της unordered\_map}

\subsection{Παράδειγμα 3 - Bloom Filters}


\section{Ασκήσεις}
\begin{enumerate}
\item α
\item β
\end{enumerate}

\begin{thebibliography}{9}

\end{thebibliography}



% Εργαστήριο 8
\chapter{Γραφήματα}
\section{Εισαγωγή}
Τα γραφήματα είναι δομές δεδομένων που συναντώνται συχνά κατά την επίλυση προβλημάτων. Η ευχέρεια προγραμματισμού αλγορίθμων που εφαρμόζονται πάνω σε γραφήματα είναι ουσιώδης. Καθώς μάλιστα συχνά ανακύπτουν προβλήματα για τα οποία έχουν διατυπωθεί αλγόριθμοι αποδοτικής επίλυσής τους η γνώση των αλγορίθμων αυτών αποδεικνύεται ισχυρός σύμμαχος στην επίλυση δύσκολων προβλημάτων. 

\section{Γραφήματα}
Ένα γράφημα ή γράφος (graph) είναι ένα σύνολο από σημεία που ονομάζονται κορυφές (vertices) ή κόμβοι (nodes) για τα οποία ισχύει ότι κάποια από αυτά είναι συνδεδεμένα απευθείας μεταξύ τους με τμήματα γραμμών που ονομάζονται ακμές (edges ή arcs). Συνήθως ένα γράφημα συμβολίζεται ως $G=(V,E)$ όπου $V$ είναι το σύνολο των κορυφών και $E$ είναι το σύνολο των ακμών.

Αν οι ακμές δεν έχουν κατεύθυνση τότε το γράφημα ονομάζεται μη κατευθυνόμενο (undirected) ενώ σε άλλη περίπτωση ονομάζεται κατευθυνόμενο (directed). Ένα πλήρες γράφημα (που όλες οι κορυφές συνδέονται απευθείας με όλες τις άλλες κορυφές) έχει $\frac{|V||V-1|}{2}$ ακμές ($|V|$ είναι το πλήθος των κορυφών του γραφήματος). Αν σε κάθε ακμή αντιστοιχεί μια τιμή τότε το γράφημα λέγεται γράφημα με βάρη. Το γράφημα του σχήματος \ref{fig:undirected_graph1} είναι ένα μη κατευθυνόμενο γράφημα με βάρη.

\begin{figure}[ht]
	\centering
	\includegraphics[width=100mm]{undirected_graph1.png}
	\caption{Ένα μη κατευθυνόμενο γράφημα 6 κορυφών με βάρη στις ακμές του}
	\label{fig:undirected_graph1}
\end{figure}

\subsection{Αναπαράσταση γραφημάτων}
Δύο διαδεδομένοι τρόποι αναπαράστασης γραφημάτων είναι οι πίνακες γειτνίασης (adjacency matrices) και οι λίστες γειτνίασης (adjacency lists).

Στους πίνακες γειτνίασης διατηρείται ένας δισδιάστατος πίνακας $n \times n$ όπου $n$ είναι το πλήθος των κορυφών του γραφήματος. Για κάθε ακμή του γραφήματος που συνενώνει την κορυφή $i$ με την κορυφή $j$ εισάγεται στη θέση $i,j$ του πίνακα το βάρος της ακμής αν το γράφημα είναι με βάρη ενώ αν δεν υπάρχουν βάρη τότε εισάγεται η τιμή 1. Όλα τα υπόλοιπα στοιχεία του πίνακα λαμβάνουν την τιμή 0. Για παράδειγμα η πληροφορία του γραφήματος του σχήματος \ref{fig:undirected_graph1} διατηρείται όπως φαίνεται στον πίνακα \ref{tbl:adjacency_table}.

% Please add the following required packages to your document preamble:
% \usepackage[table,xcdraw]{xcolor}
% If you use beamer only pass "xcolor=table" option, i.e. \documentclass[xcolor=table]{beamer}
\begin{table}[ht]
	\centering
	\begin{tabular}{|
		>{\columncolor[HTML]{C0C0C0}}l |c|c|c|c|c|c|}
		\hline
		\cellcolor[HTML]{FFFFFF} & \cellcolor[HTML]{C0C0C0} A & \cellcolor[HTML]{C0C0C0} B & \cellcolor[HTML]{C0C0C0} C & \cellcolor[HTML]{C0C0C0}D & \cellcolor[HTML]{C0C0C0} E & \cellcolor[HTML]{C0C0C0} F \\ \hline
		A                        & 0                          & 2                          & 6                          & 0                         & 0                          & 0                          \\ \hline
		B                        & 2                          & 0                          & 3                          & 1                         & 0                          & 0                          \\ \hline
		C                        & 6                          & 3                          & 0                          & 4                         & 3                          & 0                          \\ \hline
		D                        & 0                          & 1                          & 4                          & 0                         & 2                          & 10                         \\ \hline
		E                        & 0                          & 0                          & 3                          & 2                         & 0                          & 5                          \\ \hline
		F                        & 0                          & 0                          & 0                          & 10                        & 5                          & 0                          \\ \hline
	\end{tabular}
	\caption{Πίνακας γειτνίασης}
    \label{tbl:adjacency_table}
\end{table}

Στις λίστες γειτνίασης διατηρούνται λίστες που περιέχουν για κάθε κορυφή όλη την πληροφορία των συνδέσεών της με τους γειτονικούς του κόμβους. Για παράδειγμα το γράφημα του σχήματος \ref{fig:undirected_graph1} μπορεί να αναπαρασταθεί με τις ακόλουθες 6 λίστες (μια ανά κορυφή). Κάθε στοιχείο της λίστας για τον κόμβο $v$ είναι ένα ζεύγος τιμών $(w,u)$ και αναπαριστά μια ακμή από τον κόμβο $v$ στον κόμβο $u$ με βάρος $w$, όπως φαίνεται στο πίνακα \ref{tbl:adjacency_list}.

\begin{table}[ht]
	\centering
	\begin{tabular}{|
		>{\columncolor[HTML]{C0C0C0}}l |c|}
		\hline
		A & (2,B), (6,C)                \\ \hline
		B & (2,A), (3,C), (1,D)         \\ \hline
		C & (6,A), (3,B), (4,D), (3,E)  \\ \hline
		D & (1,B), (4,C), (2,E), (10,F) \\ \hline
		E & (3,C), (2,D), (5,F)         \\ \hline
		F & (10,D), (5,E)               \\ \hline
	\end{tabular}
	\caption{Λίστα γειτνίασης}
	\label{tbl:adjacency_list}
\end{table}

Περισσότερα για τις αναπαραστάσεις γραφημάτων μπορούν να βρεθούν στις αναφορές \cite{g4g_graph_representations} και \cite{he_graph_representations}.

\subsection{Ανάγνωση δεδομένων γραφήματος από αρχείο}
Υπάρχουν πολλοί τρόποι με τους οποίους μπορεί να βρίσκονται διαθέσιμα τα δεδομένα ενός γραφήματος. Στη συνέχεια παρουσιάζεται μια απλή μορφή αποτύπωσης κατευθυνόμενων με βάρη γραφημάτων χρησιμοποιώντας αρχεία απλού κειμένου. Σύμφωνα με αυτή τη μορφή για κάθε κορυφή του γραφήματος καταγράφεται σε ξεχωριστή γραμμή του αρχείου κειμένου το όνομά της ακολουθούμενο από ζεύγη τιμών που αντιστοιχούν στις ακμές που ξεκινούν από τη συγκεκριμένη κορυφή. Στο κείμενο που ακολουθεί (graph1.txt) και το οποίο αφορά το γράφημα του σχήματος \ref{fig:undirected_graph1} η πρώτη γραμμή σημαίνει ότι ο κόμβος Α συνδέεται με μια ακμή με βάρος 2 με τον κόμβο B καθώς και με μια ακμή με βάρος 6 με τον κόμβο C. Ανάλογα καταγράφεται η πληροφορία ακμών και για τις άλλες κορυφές.

\lstinputlisting[style=DOS]{lab08/graph1.txt}

Η ανάγνωση του αρχείου και η δημιουργία μιας δομής με διανύσματα που περιέχει την πληροφορία του γραφήματος μπορεί να γίνει με τη συνάρτηση read\_data που δίνεται στη συνέχεια όπου fn είναι το όνομα του αρχείου. Η συνάρτηση print\_data εμφανίζει τα δεδομένα του γραφήματος. Ο κώδικας έχει ``σπάσει'' σε 3 αρχεία (graph.hpp, graph.cpp και graph\_ex1.cpp) έτσι ώστε να είναι ευκολότερη η επαναχρησιμοποίηση του.

\lstinputlisting[caption = header file με τις συναρτήσεις για ανάγνωση και εμφάνιση γραφημάτων (graph.hpp)]{lab08/graph.hpp}

\lstinputlisting[caption = source file με τις συναρτήσεις για ανάγνωση και εμφάνιση γραφημάτων (graph.cpp)]{lab08/graph.cpp}

\lstinputlisting[caption = Ανάγνωση και εκτύπωση των δεδομένων του γραφήματος του σχήματος \ref{fig:undirected_graph1} (graph\_ex1.cpp)]{lab08/graph_ex1.cpp}

Η μεταγλώττιση και η εκτέλεση του κώδικα γίνεται με τις ακόλουθες εντολές:

\lstinputlisting[style=DOS]{lab08/compile_execute1.txt}

Η δε έξοδος που παράγεται είναι η ακόλουθη:

\lstinputlisting[style=DOS]{lab08/graph_ex1.out}

\subsection{Κατευθυνόμενα ακυκλικά γραφήματα}
Τα κατευθυνόμενα ακυκλικά γραφήματα (Directed Acyclic Graphs=DAGs) είναι γραφήματα για τα οποία δεν μπορεί να εντοπιστεί διαδρομή από μια κορυφή προς την ίδια. Στο σχήμα \ref{fig:dag1} παρουσιάζεται ένα γράφημα το οποίο δεν παρουσιάζει κύκλους. Αν για παράδειγμα υπήρχε μια ακόμα ακμή από την κορυφή E προς την κορυφή A τότε πλέον το γράφημα δεν θα ήταν DAG καθώς θα υπήρχε ο κύκλος A-C-E-A.

\begin{figure}[ht]
	\centering
	\includegraphics[width=120mm]{dag1.png}
	\caption{Ένα κατευθυνόμενο ακυκλικό γράφημα (DAG)}
	\label{fig:dag1}
\end{figure}

Τα DAGs μπορούν να χρησιμοποιηθούν στη μοντελοποίηση πολλών καταστάσεων. Μπορούν για παράδειγμα να αναπαραστήσουν εργασίες που πρέπει να εκτελεστούν και για τις οποίες υπάρχουν εξαρτήσεις όπως για παράδειγμα ότι για να ξεκινήσει η εκτέλεση της εργασίας D θα πρέπει πρώτα να έχουν ολοκληρωθεί οι εργασίες B και E. 

\subsection{Σημαντικοί αλγόριθμοι γραφημάτων}
Υπάρχουν πολλοί αλγόριθμοι που εφαρμόζονται σε γραφήματα προκειμένου να επιλύσουν ενδιαφέροντα προβλήματα που ανακύπτουν σε πρακτικές εφαρμογές. Οι ακόλουθοι αλγόριθμοι είναι μερικοί από αυτούς:
\begin{itemize}[noitemsep]
	\item Αναζήτηση συντομότερων διαδρομών από μια κορυφή προς όλες τις άλλες κορυφές (Dijkstra).
	\item Εύρεση μήκους συντομότερων διαδρομών για όλα τα ζεύγη κορυφών (Floyd Warshall) \cite{pa_floyd_warshall}.
	\item Αναζήτηση κατά βάθος (Depth First Search). Είναι αλγόριθμος διάσχισης γραφήματος ο οποίος ξεκινά από έναν κόμβο αφετηρία και επισκέπτεται όλους τους άλλους κόμβους που είναι προσβάσιμοι χρησιμοποιώντας της ακμές του γραφήματος. Λειτουργεί επεκτείνοντας μια διαδρομή όσο βρίσκει νέους κόμβους τους οποίους μπορεί να επισκεφθεί. Αν δεν βρίσκει νέους κόμβους οπισθοδρομεί και διερευνά άλλα τμήματα του γραφήματος.
	\item Αναζήτηση κατά πλάτος (Breadth First Search). Αλγόριθμος διάσχισης γραφήματος που ξεκινώντας από έναν κόμβο αφετηρία επισκέπτεται τους υπόλοιπους κόμβους σε αύξουσα σειρά βημάτων από την αφετηρία. Βήματα θεωρούνται οι μεταβάσεις από κορυφή σε κορυφή.
	\item Εντοπισμός ελάχιστου συνεκτικού δένδρου (Prim, Kruskal).
	\item Τοπολογική ταξινόμηση (Topological Sort).
	\item Εντοπισμός κυκλωμάτων Euler (Eulerian circuit).
	\item Εντοπισμός ισχυρά συνδεδεμένων συνιστωσών (Stongly Connected Components).
\end{itemize} 

Στη συνέχεια θα παρουσιαστεί ο αλγόριθμος αναζήτησης των συντομότερων διαδρομών από μια κορυφή προς όλες τις άλλες κορυφές. Ο αλγόριθμος αυτός είναι γνωστός και ως αλγόριθμος του Dijkstra.
 
\section{Αλγόριθμος του Dijkstra για εύρεση συντομότερων διαδρομών}
Ο αλγόριθμος δέχεται ως είσοδο ένα γράφημα $G=(V,E)$ και μια κορυφή του γραφήματος $s$ η οποία αποτελεί την αφετηρία. Υπολογίζει για όλες τις κορυφές $v \in V$ το μήκος του συντομότερου μονοπατιού από την κορυφή $s$ στην κορυφή $v$. Για να λειτουργήσει σωστά θα πρέπει κάθε ακμή να έχει μη αρνητικό βάρος. Αν το γράφημα περιέχει ακμές με αρνητικό βάρος τότε μπορεί να χρησιμοποιηθεί ο αλγόριθμος των Bellman-Ford \cite{brilliant_bellman_ford}.

\subsection{Περιγραφή του αλγορίθμου}
Ο αλγόριθμος εντοπίζει τις συντομότερες διαδρομές προς τις κορυφές του γραφήματος σε σειρά απόστασης από την κορυφή αφετηρία. Σε κάθε βήμα του αλγορίθμου η αφετηρία και οι ακμές προς τις κορυφές για τις οποίες έχει ήδη βρεθεί συντομότερο μονοπάτι σχηματίζουν το υποδένδρο $S$ του γραφήματος. Οι κορυφές που είναι προσπελάσιμες με 1 ακμή από το υποδένδρο $S$ είναι υποψήφιες να αποτελέσουν την επόμενη κορυφή που θα εισέλθει στο υποδένδρο. Επιλέγεται μεταξύ τους η κορυφή που βρίσκεται στη μικρότερη απόσταση από την αφετηρία. Για κάθε υποψήφια κορυφή $u$ υπολογίζεται το άθροισμα της απόστασής της από την πλησιέστερη κορυφή $v$ του δένδρου συν το μήκος της συντομότερης διαδρομής από την αφετηρία $s$ προς την κορυφή $v$. Στη συνέχεια επιλέγεται η κορυφή με το μικρότερο άθροισμα. Όταν επιλεγεί η κορυφή που πρόκειται να προστεθεί στο δένδρο τότε προστίθεται στο σύνολο των κορυφών που απαρτίζουν το υποδένδρο $S$  και για κάθε μία από τις υποψήφιες κορυφές που συνδέονται με μια ακμή με την κορυφή που επιλέχθηκε ενημερώνεται η απόστασή της από το υποδένδρο εφόσον προκύψει μικρότερη τιμή.

\paragraph{Ψευδοκώδικας}
Το σύνολο $S$ περιέχει τις κορυφές για τις οποίες έχει προσδιοριστεί η συντομότερη διαδρομή από την κορυφή $s$ ενώ το διάνυσμα $d$ περιέχει τις αποστάσεις από την κορυφή $s$ \\
1. Αρχικά $S={s}$, $d_s=0$ και για όλες τις κορυφές $i \neq s, d_i=\infty$ \\
2. Μέχρι να γίνει $S=V$ \\
3. Εντοπισμός του στοιχείου $v \notin S$ με τη μικρότερη τιμή $d_v$ και προσθήκη του στο $S$ \\
4. Για κάθε ακμή από την κορυφή $v$ στην κορυφή $u$ με βάρος $w$ ενημερώνεται η τιμή $d_u$ έτσι ώστε: \\
\centerline{$d_u=min(d_u, d_v+w)$}
5. Επιστροφή στο βήμα 2.

\paragraph{Εκτέλεση του αλγορίθμου}
Στη συνέχεια ακολουθεί παράδειγμα εκτέλεσης του αλγορίθμου για το γράφημα του σχήματος \ref{fig:undirected_graph1}.

\begin{table}[ht]
	\centering
	\label{tbl:dijkstra1}
	\begin{tabular}{|c|p{5cm}|}
		\hline
		$S=\{A\},d_A=0,d_B=2,d_C=6,d_D=\infty,d_E=\infty,d_F=\infty $   & Από το $S$ μπορούμε να φτάσουμε στις κορυφές 2 και 3 με μήκος διαδρομής 2 και 6 αντίστοιχα. Επιλέγεται η κορυφή 2.                                                                                         \\ \hline
		$S=\{A,B\}, d_A=0, d_B=2, d_C=5, d_D=3, d_E=\infty, d_F=\infty$ & Από το $S$ μπορούμε να φτάσουμε στις κορυφές C και D με μήκος διαδρομής 5 και 3 αντίστοιχα. Επιλέγεται η κορυφή D.                                                                                         \\ \hline
		$S= \{A,B,D\},d_A=0,d_B=2,d_C=5,d_D=3,d_E=5,d_F=13$             & Από το $S$ μπορούμε να φτάσουμε στις κορυφές C, E και F με μήκος διαδρομής 5, 5 και 13 αντίστοιχα. Επιλέγεται (με τυχαίο τρόπο) ανάμεσα στις κορυφές C και E η κορυφή E. \\ \hline
		$S=\{A,B,D,C\},d_A=0,d_B=2,d_C=5,d_D=3,d_E=5,d_F=13$            & Από το $S$ μπορούμε να φτάσουμε στις κορυφές E και F με μήκος διαδρομής 5 και 13 αντίστοιχα. Επιλέγεται η κορυφή E.                                                                                        \\ \hline
		$S=\{A,B,D,C,E\},d_A=0,d_B=2,d_C=5,d_D=3,d_E=5,d_F=10$          & Η μοναδική κορυφή στην οποία μένει να φτάσουμε από το $S$ είναι η κορυφή F και το μήκος της συντομότερης διαδρομής από την A στην F είναι 10.                                        \\ \hline
		\multicolumn{2}{|c|}{$S=\{A,B,D,C,E,F\},d_A=0,d_B=2,d_C=5,d_D=3,d_E=5,d_F=10$ } \\ \hline
	\end{tabular}
	\caption{Αναλυτική εκτέλεση του αλγορίθμου}
\end{table}

\begin{table}[ht]
	\centering
	\label{tbl:dijkstra2}
	\begin{tabular}{|c|c|c|c|c|c|c|}
		\hline
		Σύνολο $S$  & A & B        & C        & D        & E        & F        \\ \hline
		$\{\}$            & 0 & $\infty$     & $\infty$ & $\infty$ & $\infty$ & $\infty$ \\ \hline
		$\{A\}$           & 0 & $2_A$        & $6_A$        & $\infty$ & $\infty$ & $\infty$ \\ \hline
		$\{A,B\}$         & 0 & $2_A$        & $5_B$        & $3_B$        & $\infty$ & $\infty$ \\ \hline
		$\{A,B,D\}$       & 0 & $2_A$        & $5_B$        & $3_B$        & $5_D$        & $13_D$       \\ \hline
		$\{A,B,D,C\}$     & 0 & $2_A$        & $5_B$        & $3_B$        & $5_D$        & $13_D$       \\ \hline
		$\{A,B,D,C,E\}$   & 0 & $2_A$        & $5_B$        & $3_B$        & $5_D$        & $10_E$       \\ \hline
		$\{A,B,D,C,E,F\}$ & 0 & $2_A$        & $5_B$        & $3_B$        & $5_D$        & $10_E$       \\ \hline
	\end{tabular}
	\caption{Συνοπτική εκτέλεση του αλγορίθμου}
\end{table}

Συνεπώς ισχύει ότι: 
\begin{itemize}[noitemsep]
	\item Για την κορυφή A η διαδρομή αποτελείται μόνο από τον κόμβο A και έχει μήκος 0.
	\item Για την κορυφή B η διαδρομή είναι η A-B με μήκος 2.
	\item Για την κορυφή C η διαδρομή είναι η A-B-C με μήκος 5.
	\item Για την κορυφή D η διαδρομή είναι η A-B-D με μήκος 3.
	\item Για την κορυφή E η διαδρομή είναι η A-B-D-E με μήκος 5.
	\item Για την κορυφή F η διαδρομή είναι η A-B-D-E-F με μήκος 10.
\end{itemize}

\paragraph{Απόδοση του αλγορίθμου}

Η ταχύτητα εκτέλεσης του αλγορίθμου εξαρτάται από τις δομές δεδομένων που χρησιμοποιούνται για να αναπαρασταθεί το γράφημα. Γενικά, πρόκειται για έναν εξαιρετικά γρήγορο αλγόριθμο με πολυπλοκότητα χειρότερης περίπτωσης $O(|E| log |V|)$, όπου $|E|$ είναι ο αριθμός των ακμών και $|V|$ ο αριθμός των κορυφών του γραφήματος.

\subsection{Κωδικοποίηση του αλγορίθμου}
\lstinputlisting[caption = header file (dijkstra.hpp)]{lab08/dijkstra.hpp}

\lstinputlisting[caption = source file (dijkstra.cpp)]{lab08/dijkstra.cpp}

\lstinputlisting[caption = source file (dijkstra\_ex1.cpp)]{lab08/dijkstra_ex1.cpp}

Η μεταγλώττιση και η εκτέλεση του κώδικα γίνεται με τις ακόλουθες εντολές:

\lstinputlisting[style=DOS]{lab08/compile_execute2.txt}

Η δε έξοδος που παράγεται είναι η ακόλουθη:

\lstinputlisting[style=DOS]{lab08/dijkstra_ex1.out}


\section{Παραδείγματα}

\subsection{Παράδειγμα 1}
Για το σχήμα \ref{fig:undirected_graph2} και με αφετηρία την κορυφή A συμπληρώστε τον πίνακα εκτέλεσης του αλγορίθμου για την εύρεση των συντομότερων διαδρομών του Dijkstra και καταγράψτε τις διαδρομές που εντοπίζονται από την αφετηρία προς όλες τις άλλες κορυφές.

\begin{figure}[ht]
	\centering
	\includegraphics[width=120mm]{undirected_graph2.png}
	\caption{Ένα μη κατευθυνόμενο γράφημα 8 κορυφών με βάρη στις ακμές του}
	\label{fig:undirected_graph2}
\end{figure}

Ο ακόλουθος πίνακας δείχνει την εκτέλεση του αλγορίθμου
\begin{table}[ht]
	\centering
	\label{tbl:dijkstra3}
	\begin{tabular}{|c|c|c|c|c|c|c|c|c|}
		\hline
		Σύνολο $S$            & A & B        & C        & D        & E        & F        & G        & H         \\ \hline
		$\{\}$                & 0 & $\infty$ & $\infty$ & $\infty$ & $\infty$ & $\infty$ & $\infty$ & $\infty$  \\ \hline
		$\{A\}$               & 0 & $2_A$    & $5_A$    & $\infty$ & $\infty$ & $\infty$ & $\infty$ & $\infty$  \\ \hline
		$\{A,Β\}$             & 0 & $2_A$    & $3_B$    & $\infty$ & $\infty$ & $5_B$    & $\infty$ & $\infty$  \\ \hline
		$\{A,B,C\}$           & 0 & $2_A$    & $3_B$    & $4_B$    & $7_C$    & $5_B$    & $\infty$ & $\infty$  \\ \hline
		$\{A,B,C,D\}$         & 0 & $2_A$    & $3_B$    & $4_B$    & $5_D$    & $5_B$    & $\infty$ & $\infty$  \\ \hline
		$\{A,B,C,D,E\}$       & 0 & $2_A$    & $3_B$    & $4_B$    & $5_D$    & $5_B$    & $14_E$   & $16_E$    \\ \hline			
		$\{A,B,C,D,E,F\}$     & 0 & $2_A$    & $3_B$    & $4_B$    & $5_D$    & $5_B$    & $11_F$   & $16_E$    \\ \hline			
		$\{A,B,C,D,E,F,G\}$    & 0 & $2_A$    & $3_B$    & $4_B$    & $5_D$    & $5_B$    & $11_F$   & $12_E$    \\ \hline			
		$\{A,B,C,D,E,F,G,H\}$ & 0 & $2_A$    & $3_B$    & $4_B$    & $5_D$    & $5_B$    & $11_F$   & $12_E$    \\ \hline
	\end{tabular}
	\caption{Συνοπτική εκτέλεση του αλγορίθμου}
\end{table}

Οι συντομότερες διαδρομές είναι:
\begin{itemize}[noitemsep]
\item Για την κορυφή A η διαδρομή είναι η A με μήκος 0
\item Για την κορυφή B η διαδρομή είναι η A-B με μήκος 2
\item Για την κορυφή C η διαδρομή είναι η A-B-C με μήκος 3
\item Για την κορυφή D η διαδρομή είναι η A-B-D με μήκος 4
\item Για την κορυφή E η διαδρομή είναι η A-B-D-E με μήκος 5
\item Για την κορυφή F η διαδρομή είναι η A-B-F με μήκος 5
\item Για την κορυφή G η διαδρομή είναι η A-B-F-G με μήκος 11
\item Για την κορυφή H η διαδρομή είναι η A-B-F-G-H με μήκος 12
\end{itemize}

\subsection{Παράδειγμα 2}
Γράψτε πρόγραμμα που να διαβάζει ένα γράφημα και να εμφανίζει για κάθε κορυφή το βαθμό της, δηλαδή το πλήθος των κορυφών με τις οποίες συνδέεται απευθείας καθώς και το μέσο όρο βαρών για αυτές τις ακμές. Επιπλέον για κάθε κορυφή να εμφανίζει τις υπόλοιπες κορυφές οι οποίες μπορούν να προσεγγιστούν με διαδρομές μήκους 1,2,3 κοκ.

\lstinputlisting[caption = (lab08\_ex2.cpp)]{lab08/lab08_ex2.cpp}

Η μεταγλώττιση και η εκτέλεση του κώδικα γίνεται με τις ακόλουθες εντολές:

\lstinputlisting[style=DOS]{lab08/compile_execute3.txt}

Η δε έξοδος που παράγεται είναι η ακόλουθη:

\lstinputlisting[style=DOS]{lab08/lab08_ex2.out}

\section{Ασκήσεις}
\begin{enumerate}
	\item Υλοποιήστε τον αλγόριθμο των Bellman-Ford \cite{brilliant_bellman_ford} για την εύρεση της συντομότερης διαδρομής από μια κορυφή προς όλες τις άλλες κορυφές.
	\item Υλοποιήστε έναν αλγόριθμο τοπολογικής ταξινόμησης για DAGs.
\end{enumerate}

\begin{thebibliography}{9}
\bibitem{g4g_graph_representations}
Geeks for Geeks, Graphs and its representations \href{https://www.geeksforgeeks.org/graph-and-its-representations/}{https://www.geeksforgeeks.org/graph-and-its-representations/}

\bibitem{he_graph_representations}
HackerEarth, Graph representation, \href{https://www.hackerearth.com/practice/algorithms/graphs/graph-representation/tutorial/}{https://www.hackerearth.com/practice/algorithms/graphs/graph-representation/tutorial/}

\bibitem{algorithm_visualization_dijkstra}
Algorithm visualization, Dijkstra's shortest path \href{https://www.cs.usfca.edu/~galles/visualization/Dijkstra.html}{https://www.cs.usfca.edu/~galles/visualization/Dijkstra.html}

\bibitem{brilliant_bellman_ford}
Brilliant.org, Bellman-Ford Algorithm \href{https://brilliant.org/wiki/bellman-ford-algorithm/}{https://brilliant.org/wiki/bellman-ford-algorithm/}

\bibitem{pa_floyd_warshall}
Programming-Algorithms.net, Floyd-Warshall algorithm \href{http://www.programming-algorithms.net/article/45708/Floyd-Warshall-algorithm}{http://www.programming-algorithms.net/article/45708/Floyd-Warshall-algorithm}

\end{thebibliography}


% Εργαστήριο 9
\chapter{Δένδρα}
\section{Εισαγωγή}
Τα δένδρα όπως και τα γραφήματα είναι μη γραμμικές δομές δεδομένων. Τα δένδρα επιτρέπουν ιεραρχική οργάνωση των δεδομένων όπως φαίνεται στο Σχήμα \ref{fig:binary_tree}. 

\section{Δένδρα}

Ένα δένδρο αποτελείται από κόμβους που συνδέονται μεταξύ τους με ακμές. Ο πρώτος (υψηλότερος) κόμβος του δένδρου ονομάζεται ρίζα ενώ οι κόμβοι που βρίσκονται στα άκρα του δένδρου λέγονται φύλλα. Οι κόμβοι με τους οποίους συνδέεται απευθείας ένας κόμβος ονομάζονται παιδιά του κόμβου. Αντίστοιχα, ένας κόμβος που έχει παιδιά ονομάζεται γονέας των αντίστοιχων παιδιών-κόμβων. Απόγονοι ενός κόμβου είναι οι κόμβοι για τους οποίους υπάρχει διαδρομή πραγματοποιώντας διαδοχικές μεταβάσεις από γονείς σε παιδιά. 

Τα δένδρα είναι αναδρομικές δομές από τη φύση τους. Κάθε κόμβος ενός δένδρου ορίζει έναν αριθμό από μικρότερα δένδρα, ένα για κάθε παιδί του.
 
\begin{figure}[htbp]
  \centering
  \includegraphics[width=80mm]{Binary_tree.pdf}
  \caption{Ένα απλό δένδρο \cite{wikipedia_binary_tree}}
  \label{fig:binary_tree}
\end{figure}

Ύψος δένδρου, Βάθος κορυφής

\section{Δυαδικά δένδρα}

Δυαδικό δένδρο είναι ένα δένδρο για το οποίο ισχύει ότι κάθε κόμβος έχει το πολύ δύο παιδιά \cite{parlante_binary_tree}.

\subsection{Αναζήτηση κατά βάθος}
\subsubsection{pre-order DFS}
\subsubsection{in-order DFS}
\subsubsection{post-order DFS}

\subsection{Αναζήτηση κατά πλάτος}

\lstinputlisting[caption = header file για το δυαδικό δένδρο (binary\_tree.hpp)]{lab09/binary_tree.hpp}

\lstinputlisting[caption = source file για το δυαδικό δένδρο αναζήτησης (binary\_tree.cpp)]{lab09/binary_tree.cpp}

\lstinputlisting[caption = Δοκιμή των συναρτήσεων του δυαδικού δένδρου (lab09\_ex1.cpp)]{lab09/lab09_ex1.cpp}

Η μεταγλώττιση και η εκτέλεση του κώδικα γίνεται με τις ακόλουθες εντολές:

%\lstinputlisting[style=DOS]{lab08/compile_execute2.txt}

Η δε έξοδος που παράγεται είναι η ακόλουθη:

%\lstinputlisting[style=DOS]{lab09/lab09_ex1.out}


\section{Δυαδικά δένδρα αναζήτησης}

Σε ένα δυαδικό δένδρο αναζήτησης θα πρέπει να ισχύει ότι για κάθε κόμβο Α όλες οι τιμές κλειδιών στο δένδρο αριστερά του κόμβου Α θα πρέπει να είναι μικρότερες από την τιμή κλειδιού του κόμβου Α. Αντίστοιχα, όλες οι τιμές κλειδιών στο δένδρο δεξιά του κάθε κόμβου Α θα πρέπει να είναι μεγαλύτερες από την τιμή κλειδιού του κόμβου Α.

\subsection{Υλοποίηση δυαδικού δένδρου αναζήτησης}

\lstinputlisting[caption = header file για το δυαδικό δένδρο αναζήτησης (bst.hpp)]{lab09/bst.hpp}

\lstinputlisting[caption = source file για το δυαδικό δένδρο αναζήτησης (bst.cpp)]{lab09/bst.cpp}

\lstinputlisting[caption = Δοκιμή των συναρτήσεων του δυαδικού δένδρου αναζήτησης (lab09\_ex1.cpp)]{lab09/lab09_ex2.cpp}

Η μεταγλώττιση και η εκτέλεση του κώδικα γίνεται με τις ακόλουθες εντολές:

%\lstinputlisting[style=DOS]{lab08/compile_execute2.txt}

Η δε έξοδος που παράγεται είναι η ακόλουθη:

%\lstinputlisting[style=DOS]{lab09/lab09_ex2.out}


\section{Παραδείγματα}

\subsection{Παράδειγμα 1}

Δεδομένου ενός δυαδικού δένδρου και μιας τιμής SUM ζητείται να βρεθεί το εαν υπάρχει διαδρομή από τη ρίζα του δένδρου μέχρι κάποιο κόμβο που να έχει άθροισμα κλειδιών ίσο με την τιμή SUM.

\subsection{Παράδειγμα 2}


\section{Ασκήσεις}
\begin{enumerate}
\item 
\item 
\end{enumerate}

\begin{thebibliography}{9}
\bibitem{wikipedia_binary_tree}
Wikipedia, Tree (data structure), \href{https://en.wikipedia.org/wiki/Tree_(data_structure)}{https://en.wikipedia.org/wiki/Tree\_(data\_structure)}

\bibitem{parlante_binary_tree}
Binary Trees by Nick Parlante, \href{http://cslibrary.stanford.edu/110/BinaryTrees.html}{http://cslibrary.stanford.edu/110/BinaryTrees.html}

\end{thebibliography}




\begin{versionhistory}
  \vhEntry{1.0}{17.01.2018}{Χρήστος Γκόγκος}{Σημειώσεις για το εργαστήριο του μαθήματος Δομές Δεδομένων και Αλγόριθμοι}
  \vhEntry{1.1}{21.08.2018}{Χρήστος Γκόγκος}{Σημειώσεις για το εργαστήριο του μαθήματος Δομές Δεδομένων και Αλγόριθμοι - προσθήκη κεφαλαίου 9 (δένδρα)}
\end{versionhistory}

\end{document}

