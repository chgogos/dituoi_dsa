\section{Εισαγωγή}
Στο εργαστήριο αυτό παρουσιάζονται αλγόριθμοι ταξινόμησης και αναζήτησης. Πρόκειται για μερικούς από τους σημαντικότερους αλγορίθμους στην επιστήμη των υπολογιστών. Σε πρακτικό επίπεδο ένα σημαντικό ποσοστό της επεξεργαστικής ισχύος των υπολογιστών δαπανάται στην ταξινόμηση δεδομένων η οποία διευκολύνει τις αναζητήσεις που ακολουθούν. 
  

\section{Εκτίμηση και μέτρηση του χρόνου εκτέλεσης κώδικα}
Η απόδοση ενός αλγορίθμου μπορεί να εκτιμηθεί θεωρητικά και εμπειρικά. Η θεωρητική μελέτη προσδιορίζει την ασυμπτωτική συμπεριφορά του αλγορίθμου, δηλαδή πως θα συμπεριφέρεται ο αλγόριθμος καθώς τα δεδομένα εισόδου αυξάνονται σε μέγεθος. Με αυτό τον τρόπο μπορεί να συγκριθούν οι αποδόσεις αλγορίθμων που επιτελούν το ίδιο έργο. Για παράδειγμα ένας αλγόριθμος με χρονική πολυπλοκότητα $O(nlog(n))$ αναμένεται να έχει ταχύτερο χρόνο εκτέλεσης καθώς το μέγεθος των δεδομένων εισόδου αυξάνεται από έναν αλγόριθμο με χρονική πολυπλοκότητα $O(n^2)$. Από την άλλη μεριά η εμπειρική εκτίμηση της απόδοσης ενός προγράμματος έχει να κάνει με τη χρονομέτρησή του για διάφορες περιπτώσεις δεδομένων εισόδου και τη σύγκρισή του με εναλλακτικές υλοποιήσεις προγραμμάτων. Στη συνέχεια θα παρουσιαστούν δύο τρόποι μέτρησης χρόνου εκτέλεσης κώδικα που μπορούν να εφαρμοστούν στη C++.

\subsection{Μέτρηση χρόνου εκτέλεσης κώδικα με τη συνάρτηση clock()}
Ο ακόλουθος κώδικας μετράει το χρόνο που απαιτεί ο υπολογισμός του αθροίσματος των τετραγωνικών ριζών 10.000.000 τυχαίων ακέραιων αριθμών με τιμές στο διάστημα από 0  έως 10.000. Η μέτρηση του χρόνου πραγματοποιείται με τη συνάρτηση clock() η οποία επιστρέφει τον αριθμό από clock ticks που έχουν περάσει από τη στιγμή που το πρόγραμμα ξεκίνησε την εκτέλεση του. Ο αριθμός των δευτερολέπτων που έχουν περάσει προκύπτει διαιρώντας τον αριθμό των clock ticks με τη σταθερά CLOCKS\_PER\_SEC. Αυτός ο τρόπος υπολογισμού του χρόνου εκτέλεσης έχει ``κληρονομηθεί'' στη C++ από τη C.

\lstinputlisting[caption = Μέτρηση χρόνου εκτέλεσης κώδικα(timing1.cpp)]{lab03/timing1.cpp}

\lstinputlisting[style=DOS]{lab03/timing1.out}

\subsection{Μέτρηση χρόνου εκτέλεσης κώδικα με τη χρήση του high\_resolution\_clock::time\_point}
Η C++ έχει προσθέσει νέους τρόπους μέτρησης του χρόνου εκτέλεσης προγραμμάτων. Στον ακόλουθο κώδικα παρουσιάζεται ένα παράδειγμα με χρήση time\_points.

\lstinputlisting[caption = Μέτρηση χρόνου εκτέλεσης κώδικα (timing2.cpp)]{lab03/timing2.cpp}

\lstinputlisting[style=DOS]{lab03/timing2.out}


\section{Αλγόριθμοι ταξινόμησης}
\subsection{Ταξινόμηση με εισαγωγή}
Η ταξινόμηση με εισαγωγή (\href{http://rosettacode.org/wiki/Sorting_algorithms/Insertion_sort}{insertion-sort}) λειτουργεί δημιουργώντας μια ταξινομημένη λίστα στο αριστερό άκρο των δεδομένων και επαναληπτικά τοποθετεί το στοιχείο το οποίο βρίσκεται δεξιά της ταξινομημένης λίστας στη σωστή θέση σε σχέση με τα ήδη ταξινομημένα στοιχεία. Ο αλγόριθμος ταξινόμησης με εισαγωγή καθώς και η κλήση του από κύριο πρόγραμμα για την αύξουσα ταξινόμηση ενός πίνακα 10 θέσεων παρουσιάζεται στον κώδικα που ακολουθεί.

\lstinputlisting[caption = Ο αλγόριθμος ταξινόμησης με εισαγωγή (insertion\_sort.cpp)]{lab03/insertion_sort.cpp}

\lstinputlisting[caption = sort1.cpp]{lab03/sort1.cpp}

\lstinputlisting[style=DOS]{lab03/sort1.out}

\subsection{Ταξινόμηση με συγχώνευση}
Η ταξινόμηση με συγχώνευση (merge-sort) είναι αναδρομικός αλγόριθμος και στηρίζεται στη συγχώνευση ταξινομημένων υποακολουθιών έτσι ώστε να δημιουργούνται νέες ταξινομημένες υποακολουθίες. Μια υλοποίηση του κώδικα ταξινόμησης με συγχώνευση παρουσιάζεται στη συνέχεια. 

\lstinputlisting[caption = Ο αλγόριθμος ταξινόμησης με συγχώνευση (merge\_sort.cpp)]{lab03/merge_sort.cpp}

\lstinputlisting[caption = sort2.cpp]{lab03/sort2.cpp}

\lstinputlisting[style=DOS]{lab03/sort2.out}

\subsection{Γρήγορη ταξινόμηση}
Ο κώδικας της γρήγορης ταξινόμησης  παρουσιάζεται στη συνέχεια. Πρόκειται για κώδικα ο οποίος καλείται αναδρομικά σε υποακολουθίες των δεδομένων και σε κάθε κλήση επιλέγει ένα στοιχείο (pivot) και διαχωρίζει τα υπόλοιπα στοιχεία έτσι ώστε αριστερά να είναι τα στοιχεία που είναι μικρότερα του pivot και δεξιά αυτά τα οποία είναι μεγαλύτερα. 

\lstinputlisting[caption = Ο αλγόριθμος γρήγορης ταξινόμησης (quick\_sort.cpp)]{lab03/quick_sort.cpp}

\lstinputlisting[caption = sort3.cpp]{lab03/sort3.cpp}

\lstinputlisting[style=DOS]{lab03/sort3.out}

\subsection{Ταξινόμηση κατάταξης}
ο αλγόριθμος ταξινόμησης κατάταξης (rank-sort) λειτουργεί ως εξής: Για κάθε στοιχείο του δεδομένου πίνακα a που επιθυμούμε να ταξινομήσουμε υπολογίζεται μια τιμή κατάταξης (rank). Η τιμή κατάταξης ενός στοιχείου του πίνακα είναι το πλήθος των μικρότερων από αυτό στοιχείων συν το πλήθος των ίσων με αυτό στοιχείων που έχουν μικρότερο δείκτη σε σχέση με αυτό το στοιχείο (δηλαδή βρίσκονται αριστερά του).  Δηλαδή ισχύει ότι η τιμή κατάταξης ενός στοιχείου x του πίνακα είναι ίση με το άθροισμα 2 όρων: του πλήθους των μικρότερων στοιχείων του x από όλο τον πίνακα  και του πλήθους των ίσων με το x στοιχείων που έχουν μικρότερο δείκτη σε σχέση με το x. Για παράδειγμα στην ακολουθία τιμών a=[44, 21, 78, 16, 56, 21] θα πρέπει να δημιουργηθεί ένας νέος πίνακας r = [3, 1, 5, 0, 4, 2]. Έχοντας υπολογίσει τον πίνακα r θα πρέπει τα στοιχεία του a να αντιγραφούν σε ένα νέο βοηθητικό πίνακα temp έτσι ώστε κάθε τιμή που υπάρχει στον πίνακα r να λειτουργεί ως δείκτης για το που πρέπει να τοποθετηθεί το αντίστοιχο στοιχείο του a στον πίνακα temp. Τέλος θα πρέπει να αντιγραφεί ο πίνακας temp στον πίνακα a.
Στη συνέχεια παρουσιάζεται ο κώδικας του αλγορίθμου rank-sort. Παρουσιάζονται δύο υλοποιήσεις. Η πρώτη υλοποίηση (rank\_sort) αφορά τον αλγόριθμο όπως έχει περιγραφεί παραπάνω ενώ η δεύτερη (rank\_sort\_in\_place) είναι από το βιβλίο ``Δομές Δεδομένων, Αλγόριθμοι και Εφαρμογές στη C++ του Sartaj Sahnii, Εκδόσεις Τζιόλα, 2004'' στη σελίδα 63 (πρόγραμμα 2.11) και δεν απαιτεί τη χρήση του βοηθητικού πίνακα temp, συνεπώς είναι αποδοτικότερος. 

\lstinputlisting[caption = Ο αλγόριθμος ταξινόμησης κατάταξης (sort4.cpp)]{lab03/sort4.cpp}

\lstinputlisting[style=DOS]{lab03/sort4.out}

\section{Αλγόριθμοι αναζήτησης}

\subsection{Σειριακή αναζήτηση}
Η σειριακή αναζήτηση είναι ο απλούστερος αλγόριθμος αναζήτησης. Εξετάζει τα στοιχεία ένα προς ένα στη σειρά μέχρι να βρει το στοιχείο που αναζητείται. Το πλεονέκτημα του αλγορίθμου είναι ότι μπορεί να εφαρμοστεί σε μη ταξινομημένους πίνακες.

\lstinputlisting[caption = Ο αλγόριθμος σειριακής αναζήτησης (search1.cpp)]{lab03/search1.cpp}

\lstinputlisting[style=DOS]{lab03/search1.out}

\subsection{Δυαδική αναζήτηση}
Η δυαδική αναζήτηση μπορεί να εφαρμοστεί μόνο σε ταξινομημένα δεδομένα. Διαιρεί επαναληπτικά την ακολουθία σε 2 υποακολουθίες και απορρίπτει την ακολουθία στην οποία συμπεραίνει ότι δεν μπορεί να βρεθεί το στοιχείο. 

\lstinputlisting[caption = Ο αλγόριθμος δυαδικής αναζήτησης (binary\_search.cpp)]{lab03/binary_search.cpp}

\lstinputlisting[caption = search2.cpp]{lab03/search2.cpp}

\lstinputlisting[style=DOS]{lab03/search2.out}

\subsection{Αναζήτηση με παρεμβολή}
Η αναζήτηση με παρεμβολή (interpolation-search) είναι μια παραλλαγή της δυαδικής αναζήτησης και μπορεί να εφαρμοστεί μόνο σε ταξινομημένα δεδομένα. Αντί να χρησιμοποιηθεί η τιμή 50\% για να διαχωριστούν τα δεδομένα σε 2 ισομεγέθεις λίστες (όπως συμβαίνει στη δυαδική αναζήτηση) υπολογίζεται μια τιμή η οποία εκτιμάται ότι θα οδηγήσει πλησιέστερα στο στοιχείο που αναζητείται. Αν l είναι ο δείκτης του αριστερότερου στοιχείου της ακολουθίας και r o δείκτης του δεξιότερου στοιχείου της ακολουθίας τότε υπολογίζεται ο συντελεστής 
c = (key−a[l])/(a[r]−a[l]) όπου key είναι το στοιχείο προς αναζήτηση και a είναι η ακολουθία τιμών στην οποία αναζητείται το key. Η ακολουθία των δεδομένων διαχωρίζεται με βάση τον συντελεστή c σε δύο υποακολουθίες. Η διαδικασία επαναλαμβάνεται ανάλογα με τη δυαδική αναζήτηση. Στη συνέχεια παρουσιάζεται ο κώδικας της αναζήτησης με παρεμβολή.

\lstinputlisting[caption = Ο αλγόριθμος αναζήτησης με παρεμβολή (interpolation\_search.cpp)]{lab03/interpolation_search.cpp}

\lstinputlisting[caption = search3.cpp]{lab03/search3.cpp}

\lstinputlisting[style=DOS]{lab03/search3.out}

\section{Παραδείγματα}
\subsection{Παράδειγμα 1}
Γράψτε πρόγραμμα που να συγκρίνει τους χρόνους εκτέλεσης των αλγορίθμων ταξινόμησης insertion-sort, merge-sort, quick-sort καθώς και του αλγορίθμου ταξινόμησης της βιβλιοθήκης STL (συνάρτηση sort). Η σύγκριση να αφορά τυχαία δεδομένα τύπου float με τιμές στο διάστημα από -1.000 έως 1.000. Τα μεγέθη των πινάκων που θα ταξινομηθούν να είναι 5.000, 10.000, 20.0000, 40.000, 80.000, 160.0000 και 320.000 αριθμών. 

\lstinputlisting[caption = Σύγκριση χρόνου εκτέλεσης αλγορίθμων ταξινόμησης (lab03\_ex1.cpp)]{lab03/lab03_ex1.cpp}

\lstinputlisting[style=DOS]{lab03/lab03_ex1.out}


\subsection{Παράδειγμα 2}
Γράψτε πρόγραμμα που να συγκρίνει τους χρόνους εκτέλεσης των αλγορίθμων αναζήτησης binary-search, interpolation-search και του αλγορίθμου αναζήτησης της βιβλιοθήκης STL binary\_search για ταξινομημένα ακέραια δεδομένα με τιμές στο διάστημα από 0  έως 10.000.000. Η σύγκριση να εξετάζει τα ακόλουθα μεγέθη πινάκων 5.000, 10.000, 20.000, 40.000, 80.000, 160.000 και 320.000 αριθμών. Οι χρόνοι εκτέλεσης να αφορούν τους συνολικούς χρόνους που απαιτούνται έτσι ώστε να αναζητηθούν 100.000 τυχαίες τιμές με καθένα από τους αλγορίθμους. 


\lstinputlisting[caption = Σύγκριση χρόνου εκτέλεσης αλγορίθμων ταξινόμησης (lab03\_ex2.cpp)]{lab03/lab03_ex2.cpp}

\lstinputlisting[style=DOS]{lab03/lab03_ex2.out}

\section{Ασκήσεις}
\begin{enumerate}
\item Ο αλγόριθμος bogosort αναδιατάσσει τυχαία τις τιμές ενός πίνακα μέχρι να προκύψει μια ταξινομημένη διάταξη. Γράψτε ένα πρόγραμμα που να υλοποιεί τον αλγόριθμο bogosort για την ταξινόμηση ενός πίνακα ακεραίων τιμών.  

\end{enumerate}





